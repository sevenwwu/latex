%%%%%%%%%%%%%%%%%%%%%%%%%%%%%%%%%%%%%%%%%%%%%%%%%
%%%%%%%%%%%%%%%%%%%%%%%%%%%%%%%%%%%%%%%%%%%%%%%%%
%%%      IGNORE THIS FIRST PART        %%%%%%%%%%   EXCEPT TO ENTER/DELETE YOUR NAME WHERE INDICATED
%%%%%%%%%%%%%%%%%%%%%%%%%%%%%%%%%%%%%%%%%%%%%%%%%
%%%%%%%%%%%%%%%%%%%%%%%%%%%%%%%%%%%%%%%%%%%%%%%%%

\documentclass[12pt]{amsart}
\usepackage{amsmath, amssymb, amsthm}
\usepackage{mathrsfs}
\usepackage{array}
\usepackage{enumitem}
\usepackage[usenames,dvipsnames]{xcolor}
\usepackage{graphicx}
\usepackage{fancyhdr}
\usepackage{caption}
\usepackage{hyperref}
\usepackage[labelformat=simple]{subcaption}
\usepackage[framemethod=default]{mdframed}
\usepackage{framed}
\usepackage{setspace}
\newmdenv[linecolor=NavyBlue,backgroundcolor=White]{myframe}
\renewcommand\thesubfigure{(\Alph{subfigure})}
\renewcommand\thesubfigure{(\Alph{subfigure})}

\newcounter{problem_number}[section]
\newcommand{\num}{\refstepcounter{problem_number}\arabic{problem_number}}
\newcommand{\numlabel}[1]{\refstepcounter{problem_number}\label{#1}\arabic{problem_number}}

\newtheorem*{theorem}{Theorem}
\newtheoremstyle{named}{}{}{\itshape}{}{\bfseries}{.}{.5em}{\thmnote{#3}}
\theoremstyle{named}
\newtheorem*{namedtheorem}{Theorem}

\newenvironment{prf}
{\medskip\begin{color}{Gray}\begin{framed}\begin{color}{NavyBlue}\begin{proof}[Proof]
\doublespacing}
{\end{proof}\end{color}\end{framed}\end{color}\medskip}

\newenvironment{soln}
{\begin{color}{Gray}\begin{framed}\begin{color}{NavyBlue}\begin{proof}[Solution]
\doublespacing}
{\end{proof}\end{color}\end{framed}\end{color}}

\theoremstyle{definition}
\newtheorem{problem}{Problem}

\setenumerate[1]{label=(\roman*)}

\newcommand{\jeff}[1]{\textbf{\textcolor{WildStrawberry}{#1}}}
\newcommand{\student}[1]{\textbf{\textcolor{Orange}{#1}}}
\newcommand{\peer}[1]{\textbf{\textcolor{ForestGreen}{#1}}}

\textwidth=6.5in
\hoffset-.75in
\textheight=9in
\voffset-.75in
\footskip=30pt
\headheight=14pt
\pagestyle{fancy}
\lhead{\emph{\textcolor{Gray}{Homework \#1}}}			%%  UPDATE THE VERSION HERE!!!
\rhead{\emph{\textcolor{Gray}{student-name}}}			%%  ENTER/DELETE YOUR NAME HERE!!!
\chead{\emph{\textcolor{Gray}{MATH 309}}}
\cfoot{\thepage}
\renewcommand{\headrulewidth}{0.35pt}
\renewcommand{\footrulewidth}{0.35pt}
\thispagestyle{fancy}

%%%%%%%%%%%%%%%%%%%%%%%%%%%%%%%%%%%%%%%%%%%%%%%%%
%%%%%%%%%%%%%%%%%%%%%%%%%%%%%%%%%%%%%%%%%%%%%%%%%
%%%       ADD CUSTOM COMMANDS HERE     %%%%%%%%%%
%%%%%%%%%%%%%%%%%%%%%%%%%%%%%%%%%%%%%%%%%%%%%%%%%
%%%%%%%%%%%%%%%%%%%%%%%%%%%%%%%%%%%%%%%%%%%%%%%%%

\newcommand{\F}{\mathbb F}
\newcommand{\N}{\mathbb N}
\newcommand{\Q}{\mathbb Q}
\newcommand{\R}{\mathbb R}
\renewcommand{\S}{\mathbb S}
\newcommand{\Z}{\mathbb Z}
\newcommand{\RP}{\mathbb{RP}}

\newcommand{\Ll}{\mathcal L}
\newcommand{\Pp}{\mathcal P}
\newcommand{\Rr}{\mathcal R}

\newcommand{\Line}[1]{\overleftrightarrow{#1}}
\newcommand{\Ray}[1]{\overrightarrow{#1}}

%%%%%%%%%%%%%%%%%%%%%%%%%%%%%%%%%%%%%%%%%%%%%%%%%
%%%%%%%%%%%%%%%%%%%%%%%%%%%%%%%%%%%%%%%%%%%%%%%%%
%%%       NOW THE DOCUMENT BEGINS      %%%%%%%%%%
%%%%%%%%%%%%%%%%%%%%%%%%%%%%%%%%%%%%%%%%%%%%%%%%%
%%%%%%%%%%%%%%%%%%%%%%%%%%%%%%%%%%%%%%%%%%%%%%%%%

\begin{document}

For these problems, you should justify your answers. You do not need to provide a rigorous mathematical proof, but rather an informal argument.

%%%%%%%%%%%%%%%%%%%%%%%%%%%%%%%%%%%%%%%%%%%%%%%%%%%%%%%%%%%%%%%%%%%%%%%%%%%%%%
\begin{problem}
	Let $A$ and $B$ be sets.
	\begin{enumerate}
		\item Under what conditions do we have $A\times B = B\times A$?
		\item When is it true that $|\mathscr P(A)\times\mathscr P(A)| = |\mathscr P(A\times A)|$?
		\item What can you conclude if $A-B = \varnothing$?
		\item Describe in words the set $X=(A\times A)-D$, where the subset $D\subseteq A\times A$ is given by $D = \{(a,a)\,:\, a\in A\}$.
	\end{enumerate}
\end{problem}

\begin{soln} 
    \phantom{ }
    \begin{enumerate}
        \item When, $A = B$, when $A = \varnothing$, or when $B = \varnothing$.
        Or rephrased: $A\times B = B\times A \iff  (A = B) \lor (A =
        \varnothing) \lor (B = \varnothing)$ Proof by contraposition: Assume $A
        \neq B \neq \varnothing$, where $ x \in A $ but $x \notin B$. Then,
        $\{(x,b) : b \in B \} \subseteq (A \times B)$ and  $\{(x,b) : b \in B \}
        \nsubseteq (B \times A)$. Because at least one member is present in $(A
        \times B)$ and not $(B \times A)$, $(A \times B) \neq (B \times A)$. The
        other cases are trivial: if $A = B$, then $A \times B = A \times A = B
        \times A$. The cartesian product of any set and the emptyset is the
        emptyset, because their are no elements to iterate over. 
        \item $|\mathscr P(A)\times\mathscr P(A)| = |\mathscr P(A\times A)|$ 
        
        \noindent$|\mathscr P(A)| \cdot |\mathscr P(A)| = 2^{(|A| \cdot |A|)}$

        \noindent$2^{|A|} \cdot 2^{|A|} = 2^{(|A|^{2})}$

        \noindent$2^{2|A|} = 2^{(|A|^{2})}$

        \noindent$2|A| = |A|^2$

        \noindent$0 = |A|^2 - 2|A| = |A|(|A| - 2)$


        
        \noindent Thus, $|A| \in \{0,2\}$

        \item $ A - B = \{x : x \in A, x \notin B\} = \varnothing$
        
        No $x$ in $A$ that's not in $B$, thus, $\forall x(\lnot(x \in B)
        \rightarrow \lnot(x \in A))$. Applying logical inference of
        contraposition: $\forall x((x \in A) \rightarrow (x \in B))$. For every
        $x$ in $A$, it is also in $B$. This is the definition of a subset. Thus,
        $A \subseteq B$. 

        \item $X$ is the emptyset. $D = \{(a,a) : a \in A\}$ is the simplified
        definition of the cartesian product if the left and right hand arguments
        are the same. $D = (A \times A)$. $(A \times A) - D \iff (A \times A) -
        (A \times A)$. If we take the conclusion of the above item, then,
        because  $(A \times A) \subseteq (A \times A)$, $(A \times A) - (A
        \times A) = \varnothing = X$.
    \end{enumerate}
	
\end{soln}
%%%%%%%%%%%%%%%%%%%%%%%%%%%%%%%%%%%%%%%%%%%%%%%%%%%%%%%%%%%%%%%%%%%%%%%%%%%%%%
\begin{problem}
	Determine whether each of the following is true or false; justify your answer.
	\begin{enumerate}
		\item $\R^2\subseteq\R^3$
		\item $A\times\varnothing = \varnothing$ for every set $A$.
		\item If $A\subseteq B$, then $\mathscr P(A)\subseteq\mathscr P(B)$.	
		\item If $\mathscr P(A)\subseteq\mathscr P(B)$, then $A\subseteq B$. 
	\end{enumerate}
	
\end{problem}

\begin{soln}
    \phantom{ }
    \begin{enumerate}
        \item $\R^2\subseteq\R^3$ is false because there is at least one element
        in $\R^2$ (take the ordered pair $(0,0)$ for example), that is not in
        $\R^3$ (which has a similar but different math object $(0,0,0)$). 

        \item $A\times\varnothing = \varnothing$ for every set $A$ is true and
        was demonstrated above. ``The cartesian product of any set and the
        emptyset is the emptyset, because their are no elements to iterate
        over.'' Or rephrased, there are no pairs of elements we can make because
        $\varnothing$ has no elements.

        \item If $A\subseteq B$, then $\mathscr P(A)\subseteq\mathscr P(B)$ is
        true. $\mathscr P(A) = \{X : X \subseteq A\}$. $X \subseteq A \subseteq
        B \implies X \subseteq B$. $\mathscr P(A)$ thus contains only subsets of
        B. Because, $\mathscr P(B)$ contains \textit{all} subsets of B,
        $\mathscr P(A)\subseteq \mathscr P(B)$

        \item If $\mathscr P(A)\subseteq\mathscr P(B)$, then $A\subseteq B$ is true.
        
        \noindent Contraposition: $A \nsubseteq B \implies \mathscr P(A) \nsubseteq \mathscr P(B)$.
        Let $x \in A$ and $x \notin B$. Then, $\{x\} \in \mathscr P(A)$ and $\{x\} \notin
        \mathscr P(B)$. Thus, $\mathscr P(A) \nsubseteq \mathscr P(B)$. 

        \phantom{ }

        \noindent Direct: $\{X : X \subseteq A\} \subseteq \mathscr P(B)$, thus, each $X
        \in \mathscr P(B)$ including where $X = A$. Because $\mathscr P(B)$ contains
        $A$ as an element, $A$ must be one combination of all the elements in $B$. 
        This means $\forall x ((x \in A) \rightarrow (x \in B))$ and is the definition of
        a subset. 
	\end{enumerate}
\end{soln}

\end{document}