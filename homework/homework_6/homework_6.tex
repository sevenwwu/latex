%%%%%%%%%%%%%%%%%%%%%%%%%%%%%%%%%%%%%%%%%%%%%%%%%
%%%%%%%%%%%%%%%%%%%%%%%%%%%%%%%%%%%%%%%%%%%%%%%%%
%%%      IGNORE THIS FIRST PART        %%%%%%%%%%   EXCEPT TO ENTER/DELETE YOUR NAME WHERE INDICATED
%%%%%%%%%%%%%%%%%%%%%%%%%%%%%%%%%%%%%%%%%%%%%%%%%
%%%%%%%%%%%%%%%%%%%%%%%%%%%%%%%%%%%%%%%%%%%%%%%%%

\documentclass[12pt]{amsart}
\usepackage{amsmath, amssymb, amsthm}
\usepackage{mathrsfs}
\usepackage{array}
\usepackage{enumitem}
\usepackage[usenames,dvipsnames]{xcolor}
\usepackage{graphicx}
\usepackage{fancyhdr}
\usepackage{caption}
\usepackage{hyperref}
\usepackage[labelformat=simple]{subcaption}
\usepackage[framemethod=default]{mdframed}
\usepackage{framed}
\usepackage{setspace}
\usepackage{changepage}

\newmdenv[linecolor=NavyBlue,backgroundcolor=White]{myframe}
\renewcommand\thesubfigure{(\Alph{subfigure})}
\renewcommand\thesubfigure{(\Alph{subfigure})}

\newcounter{problem_number}[section]
\newcommand{\num}{\refstepcounter{problem_number}\arabic{problem_number}}
\newcommand{\numlabel}[1]{\refstepcounter{problem_number}\label{#1}\arabic{problem_number}}

\newtheorem*{theorem}{Theorem}
\newtheoremstyle{named}{}{}{\itshape}{}{\bfseries}{.}{.5em}{\thmnote{#3}}
\theoremstyle{named}
\newtheorem*{namedtheorem}{Theorem}

\newenvironment{prf}
{\medskip\begin{color}{Gray}\begin{framed}\begin{color}{NavyBlue}\begin{proof}[Proof]
\doublespacing}
{\end{proof}\end{color}\end{framed}\end{color}\medskip}

\newenvironment{soln}
{\begin{color}{Gray}\begin{framed}\begin{color}{NavyBlue}\begin{proof}[Solution]
\doublespacing}
{\end{proof}\end{color}\end{framed}\end{color}}

\theoremstyle{definition}
\newtheorem{problem}{Problem}

\setenumerate[1]{label=(\roman*)}

\newcommand{\jeff}[1]{\textbf{\textcolor{WildStrawberry}{#1}}}
\newcommand{\student}[1]{\textbf{\textcolor{Orange}{#1}}}
\newcommand{\peer}[1]{\textbf{\textcolor{ForestGreen}{#1}}}

\textwidth=6.5in
\hoffset-.75in
\textheight=9in
\voffset-.75in
\footskip=30pt
\headheight=14pt
\pagestyle{fancy}
\lhead{\emph{\textcolor{Gray}{Homework \#6}}}			%%  UPDATE THE VERSION HERE!!!
\rhead{\emph{\textcolor{Gray}{}}}			%%  ENTER/DELETE YOUR NAME HERE!!!
\chead{\emph{\textcolor{Gray}{MATH 309}}}
\cfoot{\thepage}
\renewcommand{\headrulewidth}{0.35pt}
\renewcommand{\footrulewidth}{0.35pt}
\thispagestyle{fancy}

%%%%%%%%%%%%%%%%%%%%%%%%%%%%%%%%%%%%%%%%%%%%%%%%%
%%%%%%%%%%%%%%%%%%%%%%%%%%%%%%%%%%%%%%%%%%%%%%%%%
%%%       ADD CUSTOM COMMANDS HERE     %%%%%%%%%%
%%%%%%%%%%%%%%%%%%%%%%%%%%%%%%%%%%%%%%%%%%%%%%%%%
%%%%%%%%%%%%%%%%%%%%%%%%%%%%%%%%%%%%%%%%%%%%%%%%%

\newcommand{\F}{\mathbb F}
\newcommand{\N}{\mathbb N}
\newcommand{\Q}{\mathbb Q}
\newcommand{\R}{\mathbb R}
\renewcommand{\S}{\mathbb S}
\newcommand{\Z}{\mathbb Z}
\newcommand{\RP}{\mathbb{RP}}

\newcommand{\Ll}{\mathcal L}
\newcommand{\Pp}{\mathcal P}
\newcommand{\Rr}{\mathcal R}

\newcommand{\Line}[1]{\overleftrightarrow{#1}}
\newcommand{\Ray}[1]{\overrightarrow{#1}}

%%%%%%%%%%%%%%%%%%%%%%%%%%%%%%%%%%%%%%%%%%%%%%%%%
%%%%%%%%%%%%%%%%%%%%%%%%%%%%%%%%%%%%%%%%%%%%%%%%%
%%%       NOW THE DOCUMENT BEGINS      %%%%%%%%%%
%%%%%%%%%%%%%%%%%%%%%%%%%%%%%%%%%%%%%%%%%%%%%%%%%
%%%%%%%%%%%%%%%%%%%%%%%%%%%%%%%%%%%%%%%%%%%%%%%%%

\begin{document}

For these problems, you should justify your answers. You do not need to provide a rigorous mathematical proof, but rather an informal argument.

\vspace{5mm}

%%%%%%%%%%%%%%%%%%%%%%%%%%%%%%%%%%%%%%%%%%%%%%%%%%%%%%%%%%%%%%%%%%%%%%%%%%%%%%
\begin{problem}
	Give a combinatorial proof of the fact that ${n\choose k}{k\choose m} = {n\choose m}{n-m\choose k-m}$.
\end{problem}

\begin{soln}
    \phantom{ }

	\noindent The left side of the equality reads ``Pick $k$ items from a set of $n$ items, then mark/color/indicate $m$
    of those picked''.

    \noindent The right side of the equality reads ``Pick $m$ items from a set of $n$ items and mark them. Then, from the
    remaining $n-m$ items, fill the selection by picking $k$ items total (which is $k-m$ more).''

    \noindent Both sides are really counting the same thing, we are just doing the same commutative actions in a different order. 

    % \noindent $\displaystyle\frac{n!}{k!(n-k)!}\frac{k!}{m!(k-m)!} = \frac{n!}{m!(n-m)!}\frac{(n-m)!}{(k-m)!(n-m-k+m)!}$

    % \noindent $\displaystyle=\frac{n!}{m!(n-m)!}\frac{(n-m)!}{(k-m)!(n-k)!}$

    % \noindent $\displaystyle=\frac{n!}{(n-k)!}\frac{}{m!(k-m)!}$

    % \noindent $\displaystyle=\frac{n!}{(n-k)!}\frac{}{m!(k-m)!} \cdot \frac{k!}{k!}$

    % \noindent $\displaystyle=\frac{n!}{k!(n-k)!}\frac{k!}{m!(k-m)!}$

\end{soln}

%%%%%%%%%%%%%%%%%%%%%%%%%%%%%%%%%%%%%%%%%%%%%%%%%%%%%%%%%%%%%%%%%%%%%%%%%%%%%%
\begin{problem}
	Give a combinatorial proof of the fact that $\displaystyle \sum_{k=0}^n2^k{n\choose k} = 3^n$.
	Then, give a proof using the binomial theorem.
\end{problem}

\begin{soln}
    \phantom{ }

    \textbf{Combinatorial Proof:}

    \begin{adjustwidth}{2em}{0em}
        \noindent The left side the equality is counting the number of different ways we can 
        put $n$ elements into $2$ exclusive subsets $A$ and $B$, where we do not have to use all $n$ elements.
        For example with $1$ element, we can put it in into $A$ or $B$. With $2$ elements $x$ and $y$,
        we can put them both in $A$, both in $B$, $x$ in $A$ and $y$ in $B$, or $x$ in $B$ and $y$ in $A$.

        \phantom{ }

        \noindent The right side of the equality is counting
        the number of different ways we can put $n$ objects into $3$ bins. We have three choices
        for each element, with a unique thing to count every time we move an element between bins.

        \phantom{ }

        \noindent Now, this is actually really what we're doing on the left side. For each element
        it is either being picked to be placed in subset $A$ or $B$, or it is being left out of both.
        This left out of both can simply be subset $C$. $A \cup B \cup C$ contain all elements, and
        $A \cap B = \varnothing$, $B \cap C = \varnothing$, $A \cap C = \varnothing$. This is
        exactly what we are doing on the right side. 
    \end{adjustwidth}

    \phantom{ }

    \textbf{Binomial Theorem Proof:}

    \begin{adjustwidth}{2em}{0em}
        $\displaystyle\sum_{k = 0}^{n} {n \choose k} x^{n-k}y^k = (x + y)^n$

        \phantom{ }

        \noindent Looking at the given equality, $\displaystyle \sum_{k=0}^n2^k{n\choose k}$ is
        very similar to the binomial theorem's left side. 
        
        \noindent Setting $x = 1$ and $y = 2$, the binomial theorem shows that:

        \noindent $\displaystyle\sum_{k = 0}^{n} 2^k {n \choose k} = (1 + 2)^n = 3^n$.
    \end{adjustwidth}

\end{soln}

%%%%%%%%%%%%%%%%%%%%%%%%%%%%%%%%%%%%%%%%%%%%%%%%%%%%%%%%%%%%%%%%%%%%%%%%%%%%%%


\end{document}