%%%%%%%%%%%%%%%%%%%%%%%%%%%%%%%%%%%%%%%%%%%%%%%%%
%%%%%%%%%%%%%%%%%%%%%%%%%%%%%%%%%%%%%%%%%%%%%%%%%
%%%      IGNORE THIS FIRST PART        %%%%%%%%%%   EXCEPT TO ENTER/DELETE YOUR NAME WHERE INDICATED
%%%%%%%%%%%%%%%%%%%%%%%%%%%%%%%%%%%%%%%%%%%%%%%%%
%%%%%%%%%%%%%%%%%%%%%%%%%%%%%%%%%%%%%%%%%%%%%%%%%

\documentclass[12pt]{amsart}
\usepackage{amsmath, amssymb, amsthm}
\usepackage{mathrsfs}
\usepackage{array}
\usepackage{enumitem}
\usepackage[usenames,dvipsnames]{xcolor}
\usepackage{graphicx}
\usepackage{fancyhdr}
\usepackage{caption}
\usepackage{hyperref}
\usepackage[labelformat=simple]{subcaption}
\usepackage[framemethod=default]{mdframed}
\usepackage{framed}
\usepackage{setspace}
\usepackage{changepage}

\setlength{\parindent}{0pt}


\newmdenv[linecolor=NavyBlue,backgroundcolor=White]{myframe}
\renewcommand\thesubfigure{(\Alph{subfigure})}
\renewcommand\thesubfigure{(\Alph{subfigure})}

\newcounter{problem_number}[section]
\newcommand{\num}{\refstepcounter{problem_number}\arabic{problem_number}}
\newcommand{\numlabel}[1]{\refstepcounter{problem_number}\label{#1}\arabic{problem_number}}

\newtheorem*{theorem}{Theorem}
\newtheoremstyle{named}{}{}{\itshape}{}{\bfseries}{.}{.5em}{\thmnote{#3}}
\theoremstyle{named}
\newtheorem*{namedtheorem}{Theorem}

\newenvironment{prf}
{\medskip\begin{color}{Gray}\begin{framed}\begin{color}{NavyBlue}\begin{proof}[Proof]
\doublespacing}
{\end{proof}\end{color}\end{framed}\end{color}\medskip}

\newenvironment{soln}
{\begin{color}{Gray}\begin{framed}\begin{color}{NavyBlue}\begin{proof}[Solution]
\doublespacing}
{\end{proof}\end{color}\end{framed}\end{color}}

\theoremstyle{definition}
\newtheorem{proposition}{Proposition}
\newtheorem{problem}[proposition]{Problem}

\setenumerate[1]{label=(\roman*)}

\newcommand{\jeff}[1]{\textbf{\textcolor{WildStrawberry}{#1}}}
\newcommand{\student}[1]{\textbf{\textcolor{Orange}{#1}}}
\newcommand{\peer}[1]{\textbf{\textcolor{ForestGreen}{#1}}}

\textwidth=6.5in
\hoffset-.75in
\textheight=9in
\voffset-.75in
\footskip=30pt
\headheight=14pt
\pagestyle{fancy}
\lhead{\emph{\textcolor{Gray}{Homework \#10}}}			%%  UPDATE THE VERSION HERE!!!
\rhead{\emph{\textcolor{Gray}{student-name}}}			%%  ENTER/DELETE YOUR NAME HERE!!!
\chead{\emph{\textcolor{Gray}{MATH 309}}}
\cfoot{\thepage}
\renewcommand{\headrulewidth}{0.35pt}
\renewcommand{\footrulewidth}{0.35pt}
\thispagestyle{fancy}

%%%%%%%%%%%%%%%%%%%%%%%%%%%%%%%%%%%%%%%%%%%%%%%%%
%%%%%%%%%%%%%%%%%%%%%%%%%%%%%%%%%%%%%%%%%%%%%%%%%
%%%       ADD CUSTOM COMMANDS HERE     %%%%%%%%%%
%%%%%%%%%%%%%%%%%%%%%%%%%%%%%%%%%%%%%%%%%%%%%%%%%
%%%%%%%%%%%%%%%%%%%%%%%%%%%%%%%%%%%%%%%%%%%%%%%%%

\newcommand{\F}{\mathbb F}
\newcommand{\N}{\mathbb N}
\newcommand{\Q}{\mathbb Q}
\newcommand{\R}{\mathbb R}
\renewcommand{\S}{\mathbb S}
\newcommand{\Z}{\mathbb Z}
\newcommand{\RP}{\mathbb{RP}}

\newcommand{\Ll}{\mathcal L}
\newcommand{\Pp}{\mathcal P}
\newcommand{\Rr}{\mathcal R}

\newcommand{\Line}[1]{\overleftrightarrow{#1}}
\newcommand{\Ray}[1]{\overrightarrow{#1}}

%%%%%%%%%%%%%%%%%%%%%%%%%%%%%%%%%%%%%%%%%%%%%%%%%
%%%%%%%%%%%%%%%%%%%%%%%%%%%%%%%%%%%%%%%%%%%%%%%%%
%%%       NOW THE DOCUMENT BEGINS      %%%%%%%%%%
%%%%%%%%%%%%%%%%%%%%%%%%%%%%%%%%%%%%%%%%%%%%%%%%%
%%%%%%%%%%%%%%%%%%%%%%%%%%%%%%%%%%%%%%%%%%%%%%%%%

\begin{document}

%%%%%%%%%%%%%%%%%%%%%%%%%%%%%%%%%%%%%%%%%%%%%%%%%%%%%%%%%%%%%%%%%%%%%%%%%%%%%%
\begin{problem}
	Consider the function $f\colon [0,2\pi]\to [-1,1]$ given by $f(x) = \cos x$. Determine each of the following sets.
	\begin{enumerate}
		\item $f\left([0,\pi]\right)$
		\item $f\left(\{\pi\}\right)$
		\item $f\left((0,\frac{\pi}{2})\right)$
		\item $f\left((0,\pi)\right)$
		\item $f^{-1}\left(\{-1,1\}\right) = \{0,\pi,2\pi\}$
		\item $f^{-1}\left(\{0,1\}\right)$
		\item $f^{-1}\left((-1,0)\right)$
		\item $f^{-1}\left(\{0\}\right)$
	\end{enumerate}
\end{problem}

\begin{soln}
    \phantom{ }

    \begin{enumerate}
        \item $[-1,1]$
        \item $\{-1\}$
        \item $(0,1)$
        \item $(-1,1)$
        \item $\{0,\pi,2\pi\}$
        \item $\{\frac{\pi}{2},\pi,\frac{3\pi}{2},2\pi\}$
        \item $(\frac{\pi}{2},\frac{3\pi}{2})$
        \item $\{\frac{\pi}{2},\frac{3\pi}{2}\}$
    \end{enumerate}
\end{soln}
%%%%%%%%%%%%%%%%%%%%%%%%%%%%%%%%%%%%%%%%%%%%%%%%%%%%%%%%%%%%%%%%%%%%%%%%%%%%%%

\vspace{20em}

\begin{problem}
	Consider $f\colon A\to B$.
	\begin{enumerate}
		\item Prove $f$ is injective if and only if $X = f^{-1}(f(X))$ for all $X\subseteq A$.
		\item Prove $f$ is surjective if and only if $Y=f(f^{-1}(Y))$ for all $Y\subseteq B$.
	\end{enumerate}
\end{problem}

\begin{soln}
	\phantom{ }
    
    \begin{enumerate}
        \item 
        \phantom{ }

        \noindent Suppose $X \subseteq A$.

        \noindent Suppose the function $f$ is injective with a range of $B'$.

        \noindent Consider the image of $f_\text{img} : \mathscr P(A) \rightarrow \mathscr P(B')$
        where 
        
        \noindent $f_\text{img} = \{(X,\{f(x) \in B' : x \in X\}) \in \mathscr P(A) \times \mathscr P(B')\}$.

        \noindent This relationship is a function because for every subset $X$ of $A$
        there is exactly one related subset of $B'$, particularly $\{f(x) \in B' : x \in X\}$.

        \noindent $f_\text{img}$ is also injective, as seen below:

        \begin{adjustwidth}{2em}{0em}
            \noindent Assume $f_\text{img}$ is not injective.

            \noindent Then, for some $M$ and $N$, $f_\text{img}(M) = f_\text{img}(N)$ and $M \neq N$.

            \noindent Then $\{f(x) \in B' : x \in M\} = \{f(x) \in B' : x \in N\}$.

            \noindent Let $m \in M$ and $x \notin N$ without loss of generality.

            \noindent Because $f$ is injective, $f(m)$ is only in $f_\text{img}(M)$
            and not $f_\text{img}(N)$.

            \noindent So, $f_\text{img}(M) \neq f_\text{img}(N)$, which is a contradiction.
        \end{adjustwidth}

        \noindent $f_\text{img}$ is also surjective:

        \begin{adjustwidth}{2em}{0em}
            Let $M \in \mathscr P(B')$.

            \noindent Then $M = f_\text{img}(\{x \in A : f(x) \in M\})$.
        \end{adjustwidth}

        \noindent So $f_\text{img}$ is bijective. 

        \noindent Recall that bijective functions are invertible, so $f^{-1}_{\text{img}} : 
        \mathscr P(B') \rightarrow \mathscr P(A)$

        \noindent So, the images composed: $f^{-1}_\text{img} \circ f_\text{img} = i_{\mathscr P(A)}$.
        
        \noindent Thus, $f^{-1}(f(X)) = X$.

        \phantom{ }

        \noindent Suppose that $X = f^{-1}(f(X))$ for all $X \subseteq A$.

        \noindent So, for all singletons $\{x\} \subseteq A$, $\{x\} = f^{-1}(f(\{x\}))$.

        \noindent Assume there exists some $y,z \in A$ such that $f(y) = f(z)$ and $y \neq z$. 

        \noindent So, the set $f^{-1}(f(\{y\}))$ contains both $y$ and $z$. 

        \noindent This is a contradiction. So, if $f(y) = f(z)$, then $y = z$, for all $y,z \in A$.

        \noindent Thus, $f$ is injective.     

        \item
        \phantom{ }
        
        \noindent Suppose $Y \subseteq B$.

        \noindent Suppose the function $f$ is surjective. 

        \begin{adjustwidth}{2em}{0em}
            \noindent Suppose $y \in Y$.

            \noindent Because $f$ is surjective, there is some $x_y$ for which $f(x_y) = y$.
    
            \noindent So, $x_y \in \{x : f(x) \in Y\}$ and $\{x : f(x) \in Y\} = f^{-1}(Y)$.
    
            \noindent So, $x_y \in f^{-1}(Y)$. 
    
            \noindent Then $y \in \{f(x) : x \in f^{-1}(Y)\}$ and $\{f(x) : x \in f^{-1}(Y)\} = f(f^{-1}(Y))$.
    
            \noindent So, $y \in f(f^{-1}(Y))$.

            \phantom{ }

            \noindent Suppose $y \in f(f^{-1}(Y))$. 

            \noindent Then $y \in \{f(x) : x \in \{x : f(x) \in Y\}\}$.

            \noindent So, there is some $x_y$ where $f(x_y) = y$ and $x_y \in \{x : f(x) \in Y\}$. 

            \noindent So, $f(x_y) \in Y$, thus $y \in Y$. 
        \end{adjustwidth}

        \phantom{ }

        \noindent Therefore, $Y = f(f^{-1}(Y))$ because an arbitrary element $y$ in either set is always in both sets. 

        \phantom{ }

        \phantom{ }

        \noindent Suppose $Y = f(f^{-1}(Y))$ for all $Y \subseteq B$.

        \noindent So, for all singletons $\{y\} \subseteq B$, $\{y\} = f(f^{-1}(\{y\}))$.

        \noindent Therefore, for every $y \in B$, there is some $x \in A$ such that $f(x) = y$. 


    \end{enumerate}
\end{soln}
%%%%%%%%%%%%%%%%%%%%%%%%%%%%%%%%%%%%%%%%%%%%%%%%%%%%%%%%%%%%%%%%%%%%%%%%%%%%%%


\end{document}