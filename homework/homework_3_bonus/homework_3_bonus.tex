%%%%%%%%%%%%%%%%%%%%%%%%%%%%%%%%%%%%%%%%%%%%%%%%%
%%%%%%%%%%%%%%%%%%%%%%%%%%%%%%%%%%%%%%%%%%%%%%%%%
%%%      IGNORE THIS FIRST PART        %%%%%%%%%%   EXCEPT TO ENTER/DELETE YOUR NAME WHERE INDICATED
%%%%%%%%%%%%%%%%%%%%%%%%%%%%%%%%%%%%%%%%%%%%%%%%%
%%%%%%%%%%%%%%%%%%%%%%%%%%%%%%%%%%%%%%%%%%%%%%%%%

\documentclass[12pt]{amsart}
\usepackage{amsmath, amssymb, amsthm}
\usepackage{mathrsfs}
\usepackage{array}
\usepackage{enumitem}
\usepackage[usenames,dvipsnames]{xcolor}
\usepackage{graphicx}
\usepackage{fancyhdr}
\usepackage{caption}
\usepackage{hyperref}
\usepackage[labelformat=simple]{subcaption}
\usepackage[framemethod=default]{mdframed}
\usepackage{framed}
\usepackage{verbatim}
\usepackage{setspace}
\newmdenv[linecolor=NavyBlue,backgroundcolor=White]{myframe}
\renewcommand\thesubfigure{(\Alph{subfigure})}
\renewcommand\thesubfigure{(\Alph{subfigure})}

\newcounter{problem_number}[section]
\newcommand{\num}{\refstepcounter{problem_number}\arabic{problem_number}}
\newcommand{\numlabel}[1]{\refstepcounter{problem_number}\label{#1}\arabic{problem_number}}

\newtheorem*{theorem}{Theorem}
\newtheoremstyle{named}{}{}{\itshape}{}{\bfseries}{.}{.5em}{\thmnote{#3}}
\theoremstyle{named}
\newtheorem*{namedtheorem}{Theorem}

\newenvironment{prf}
{\medskip\begin{color}{Gray}\begin{framed}\begin{color}{NavyBlue}\begin{proof}[Proof]
\doublespacing}
{\end{proof}\end{color}\end{framed}\end{color}\medskip}

\newenvironment{soln}
{\begin{color}{Gray}\begin{framed}\begin{color}{NavyBlue}\begin{proof}[Solution]
\doublespacing}
{\end{proof}\end{color}\end{framed}\end{color}}

\theoremstyle{definition}
\newtheorem{problem}{Problem}

\setenumerate[1]{label=(\roman*)}

\newcommand{\jeff}[1]{\textbf{\textcolor{WildStrawberry}{#1}}}
\newcommand{\student}[1]{\textbf{\textcolor{Orange}{#1}}}
\newcommand{\peer}[1]{\textbf{\textcolor{ForestGreen}{#1}}}

\textwidth=6.5in
\hoffset-.75in
\textheight=9in
\voffset-.75in
\footskip=30pt
\headheight=14pt
\pagestyle{fancy}
\lhead{\emph{\textcolor{Gray}{Chapter 3 Bonus Homework}}}			%%  UPDATE THE VERSION HERE!!!
\rhead{\emph{\textcolor{Gray}{Seven Lewis}}}			%%  ENTER/DELETE YOUR NAME HERE!!!
\chead{\emph{\textcolor{Gray}{MATH 309}}}
\cfoot{\thepage}
\renewcommand{\headrulewidth}{0.35pt}
\renewcommand{\footrulewidth}{0.35pt}
\thispagestyle{fancy}

%%%%%%%%%%%%%%%%%%%%%%%%%%%%%%%%%%%%%%%%%%%%%%%%%
%%%%%%%%%%%%%%%%%%%%%%%%%%%%%%%%%%%%%%%%%%%%%%%%%
%%%       ADD CUSTOM COMMANDS HERE     %%%%%%%%%%
%%%%%%%%%%%%%%%%%%%%%%%%%%%%%%%%%%%%%%%%%%%%%%%%%
%%%%%%%%%%%%%%%%%%%%%%%%%%%%%%%%%%%%%%%%%%%%%%%%%

\newcommand{\F}{\mathbb F}
\newcommand{\N}{\mathbb N}
\newcommand{\Q}{\mathbb Q}
\newcommand{\R}{\mathbb R}
\renewcommand{\S}{\mathbb S}
\newcommand{\Z}{\mathbb Z}
\newcommand{\RP}{\mathbb{RP}}

\newcommand{\Ff}{\mathcal F}
\newcommand{\Ll}{\mathcal L}
\newcommand{\Pp}{\mathcal P}
\newcommand{\Rr}{\mathcal R}

\newcommand{\Line}[1]{\overleftrightarrow{#1}}
\newcommand{\Ray}[1]{\overrightarrow{#1}}

%%%%%%%%%%%%%%%%%%%%%%%%%%%%%%%%%%%%%%%%%%%%%%%%%
%%%%%%%%%%%%%%%%%%%%%%%%%%%%%%%%%%%%%%%%%%%%%%%%%
%%%       NOW THE DOCUMENT BEGINS      %%%%%%%%%%
%%%%%%%%%%%%%%%%%%%%%%%%%%%%%%%%%%%%%%%%%%%%%%%%%
%%%%%%%%%%%%%%%%%%%%%%%%%%%%%%%%%%%%%%%%%%%%%%%%%

\begin{document}



%%%%%%%%%%%%%%%%%%%%%%%%%%%%%%%%%%%%%%%%%%%%%%%%%%%%%%%%%%%%%%%%%%%%%%%%%%%%%%
\begin{problem}
	Let $\mathcal C$ be the unit circle, and let $A$ be a collection of 20 evenly spaced points on $\mathcal C$.
	A straight segment connecting two points of $A$ is called a \emph{chord}.
	How many ways are there to draw 10 chords hitting all 20 points of $A$?
\end{problem}

\begin{soln}
    \phantom{ }

	Two different approaches:

	\begin{itemize}
		\item Pick between the $\displaystyle {20 \choose 2}$ options,
		then pick between the $\displaystyle {18 \choose 2}$ options, \dots etc.
		Then because it didn't matter in what order you selected the $10$ cords, divide
		by $10!$. 
		\item Arrange the $20$ points in a list and count the number of
		permutations. Then conceptualize that their are $10$ pairs in each
		string. Because a cord is directionless, we need to remove order 
		from each pair, thus divide by $2!$ $10$ times. Additionally, it doesn't
		matter in what order the cords appear in. So also divide by $10!$. 
	\end{itemize}

	$$\displaystyle\prod_{i=1}^{10}{2\cdot i \choose 2} \cdot \frac{1}{10!} = 
	\frac{20!}{2!^{10}\cdot10!}= 654,729,075$$

\end{soln}

%%%%%%%%%%%%%%%%%%%%%%%%%%%%%%%%%%%%%%%%%%%%%%%%%%%%%%%%%%%%%%%%%%%%%%%%%%%%%%
\begin{problem}
	How many binary strings of length 10 contain somewhere within them the string \verb|10001|?
\end{problem}

\begin{soln}
    \phantom{ }

    \noindent There are $6$ different placements of \verb|10001| with four
    positions that risk double count. We can count this by placing \verb|10001|
    in each of the six positions and then fill the rest. Thus, $6 \cdot 2^5$ is
    the result before accounting for doubly count strings.

    \noindent Now, there are four positions where we double count:

	A = \verb|10001?????|

	B = \verb|?????10001|
	
	C = \verb|?10001????|

	D = \verb|????10001?|

	\noindent I will list only the non-zero intersections for brevity:

	$|A \cap B| = 1 \implies \verb|1000110001|$
	
	$|A \cap D| = 2 \implies \verb|100010001?|$

	$|B \cap C| = 2 \implies  \verb|?100010001|$

    \noindent Thus, $6 \cdot 2^5 - 2 - 2 - 1 = 187$. 

\end{soln}



%%%%%%%%%%%%%%%%%%%%%%%%%%%%%%%%%%%%%%%%%%%%%%%%%%%%%%%%%%%%%%%%%%%%%%%%%%%%%%
\begin{problem}
	Let $X=\{1,2,3,\ldots,n\}$.
	\begin{enumerate}
		\item Determine the cardinality of the set 
		$$Z_k = \{A\in\mathscr{P}(X)\colon |A|=k\}$$
		for $0\leq k\leq n$.
	\item Find a combinatorial proof showing that $\displaystyle\sum_{k=0}^nk\binom{n}{k} = n2^{n-1}$.

	\textbf{Hint:} The parts of this problem are related.
	\end{enumerate}
\end{problem}

\begin{soln}
	\phantom{ }

	\begin{enumerate}
		\item The powers set contains all $k$-combinations of some set
		where $0 \leq k \leq n$. The indexed set builder is simply picking 
		only subsets of $X$ of a particular $k$. The number of $k$-combinations
		can be expressed as ``$n$ pick $k$''.

		\noindent Thus, $\displaystyle |Z_k| = {n \choose k}$.
		
		\phantom{ }

		\item Suppose we have sets $A$ and $B$, where $A \subseteq B
		\subseteq X$ and $|A| = 1$. ($X$ is already defined above.)
		How many different ways combinations of $A$ and $B$ are there
		for some $X$?

		\phantom{ }

		\noindent One way we can count this is by summing the count for
		each cardinality of $B$ (which can be between $0$ and $n$) where we pick one element from $B$ and put into
		$A$. This consists of picking $|B|$ elements from $X$ and $1$ element from $B$.
		In symbols, this is 
		$\displaystyle \sum_{k=0}^n{n \choose k}{k\choose 1} = \sum_{k=0}^nk{n \choose k}$.

		\phantom{ }

		\noindent Another way we can count this is by first picking $1$ element from
		$X$ to fill $A$, then fill $B$ with the remaining $n-1$ elements. Now for
		each of these $n-1$ elements, we have a $2$ choices: either put into $B$
		or leave out of $B$. To account for every possibility, we can use the multiplication
		principle. In symbols this is $\displaystyle n2^{n-1}$.

		\phantom{ }

		\noindent We have counted the same set two different ways which 
		shows that their related expressions are equivalent. Thus,
		$\displaystyle\sum_{k=0}^nk{n \choose k} = \displaystyle n2^{n-1}$. 

	\end{enumerate}

\end{soln}

\end{document}