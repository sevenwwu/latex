%%%%%%%%%%%%%%%%%%%%%%%%%%%%%%%%%%%%%%%%%%%%%%%%%
%%%%%%%%%%%%%%%%%%%%%%%%%%%%%%%%%%%%%%%%%%%%%%%%%
%%%      IGNORE THIS FIRST PART        %%%%%%%%%%   EXCEPT TO ENTER/DELETE YOUR NAME WHERE INDICATED
%%%%%%%%%%%%%%%%%%%%%%%%%%%%%%%%%%%%%%%%%%%%%%%%%
%%%%%%%%%%%%%%%%%%%%%%%%%%%%%%%%%%%%%%%%%%%%%%%%%

\documentclass[12pt]{amsart}
\usepackage{amsmath, amssymb, amsthm}
\usepackage{mathrsfs}
\usepackage{array}
\usepackage{enumitem}
\usepackage[usenames,dvipsnames]{xcolor}
\usepackage{graphicx}
\usepackage{fancyhdr}
\usepackage{caption}
\usepackage{hyperref}
\usepackage[labelformat=simple]{subcaption}
\usepackage[framemethod=default]{mdframed}
\usepackage{framed}
\usepackage{setspace}
\usepackage{changepage}

\setlength{\parindent}{0pt}

\newmdenv[linecolor=NavyBlue,backgroundcolor=White]{myframe}
\renewcommand\thesubfigure{(\Alph{subfigure})}
\renewcommand\thesubfigure{(\Alph{subfigure})}

\newcounter{problem_number}[section]
\newcommand{\num}{\refstepcounter{problem_number}\arabic{problem_number}}
\newcommand{\numlabel}[1]{\refstepcounter{problem_number}\label{#1}\arabic{problem_number}}

\newtheorem*{theorem}{Theorem}
\newtheoremstyle{named}{}{}{\itshape}{}{\bfseries}{.}{.5em}{\thmnote{#3}}
\theoremstyle{named}
\newtheorem*{namedtheorem}{Theorem}

\newenvironment{prf}
{\medskip\begin{color}{Gray}\begin{framed}\begin{color}{NavyBlue}\begin{proof}[Proof]
\doublespacing}
{\end{proof}\end{color}\end{framed}\end{color}\medskip}

\newenvironment{soln}
{\begin{color}{Gray}\begin{framed}\begin{color}{NavyBlue}\begin{proof}[Solution]
\doublespacing}
{\end{proof}\end{color}\end{framed}\end{color}}

\theoremstyle{definition}
\newtheorem{proposition}{Proposition}
\newtheorem{problem}[proposition]{Problem}

\setenumerate[1]{label=(\roman*)}

\newcommand{\jeff}[1]{\textbf{\textcolor{WildStrawberry}{#1}}}
\newcommand{\student}[1]{\textbf{\textcolor{Orange}{#1}}}
\newcommand{\peer}[1]{\textbf{\textcolor{ForestGreen}{#1}}}

\textwidth=6.5in
\hoffset-.75in
\textheight=9in
\voffset-.75in
\footskip=30pt
\headheight=14pt
\pagestyle{fancy}
\lhead{\emph{\textcolor{Gray}{Homework \#8}}}			%%  UPDATE THE VERSION HERE!!!
\rhead{\emph{\textcolor{Gray}{Seven Lewis}}}			%%  ENTER/DELETE YOUR NAME HERE!!!
\chead{\emph{\textcolor{Gray}{MATH 309}}}
\cfoot{\thepage}
\renewcommand{\headrulewidth}{0.35pt}
\renewcommand{\footrulewidth}{0.35pt}
\thispagestyle{fancy}

%%%%%%%%%%%%%%%%%%%%%%%%%%%%%%%%%%%%%%%%%%%%%%%%%
%%%%%%%%%%%%%%%%%%%%%%%%%%%%%%%%%%%%%%%%%%%%%%%%%
%%%       ADD CUSTOM COMMANDS HERE     %%%%%%%%%%
%%%%%%%%%%%%%%%%%%%%%%%%%%%%%%%%%%%%%%%%%%%%%%%%%
%%%%%%%%%%%%%%%%%%%%%%%%%%%%%%%%%%%%%%%%%%%%%%%%%

\newcommand{\F}{\mathbb F}
\newcommand{\N}{\mathbb N}
\newcommand{\Q}{\mathbb Q}
\newcommand{\R}{\mathbb R}
\renewcommand{\S}{\mathbb S}
\newcommand{\Z}{\mathbb Z}
\newcommand{\RP}{\mathbb{RP}}

\newcommand{\Ll}{\mathcal L}
\newcommand{\Pp}{\mathcal P}
\newcommand{\Rr}{\mathcal R}

\newcommand{\Line}[1]{\overleftrightarrow{#1}}
\newcommand{\Ray}[1]{\overrightarrow{#1}}

%%%%%%%%%%%%%%%%%%%%%%%%%%%%%%%%%%%%%%%%%%%%%%%%%
%%%%%%%%%%%%%%%%%%%%%%%%%%%%%%%%%%%%%%%%%%%%%%%%%
%%%       NOW THE DOCUMENT BEGINS      %%%%%%%%%%
%%%%%%%%%%%%%%%%%%%%%%%%%%%%%%%%%%%%%%%%%%%%%%%%%
%%%%%%%%%%%%%%%%%%%%%%%%%%%%%%%%%%%%%%%%%%%%%%%%%

\begin{document}

Prove the following propositions. Format your proof so each step of the proof is on its own line; each line should still be a complete sentence.

%%%%%%%%%%%%%%%%%%%%%%%%%%%%%%%%%%%%%%%%%%%%%%%%%%%%%%%%%%%%%%%%%%%%%%%%%%%%%%
\begin{proposition}
	Suppose $a,b,p\in\Z$ and $p$ is prime.
	Prove that if $p|ab$, then $p|a$ or $p|b$.
	(Suggestion: Use the Proposition on page 152.)
\end{proposition}

\begin{prf}
	\phantom{ }

	Suppose that $p|ab$. 
	
	Then $px = ab$ for some integer $x$.

	Then $\text{sign}(a) \text{sign}(p) |p|x = |a|b 
	\implies |p|y = |a|b$ where $y$ is the integer

	$\text{sign}(a)\text{sign}(p)x$.

	% For the sake simplicity, assume $a$, $b$ and $p$ are positive.

	% We can do this because if an odd number of $a$, $b$, and $p$ are 
	% negative, you can simply negate the value of $x$ to maintain
	% the equality. 
	
	% Additionally, $p$ is not ever equal to $0$, and if $a = 0$ and $b = 0$, then
	% it is trivially true that $px = ab$. 

	\textbf{Proposition 7.1:} 
	\begin{adjustwidth}{2em}{0em}
		If $a,b \in \mathbb N$, then
		there exists integers $k$ and $l$ for which $\text{gcd}(
		a,b) = ak + bl$.
	\end{adjustwidth} 

	If $a = 0$, then $p|a$. Suppose $a \neq 0$.

	So, $\text{gcd}(|a|,|p|) = |a|k + |p|l$ where $k,l \in \mathbb Z$.

	Additionally, because $p$ is prime, $|a|$ is either a
	multiple of $|p|$ or only shares the trivial divisor $1$.

	In the first case, $|p|||a| \implies p|a$, so assume $\text{gcd}(|a|,|p|) = 1$.

	Then $|a|k + |p|l = 1$.

	Multiplying by $b$: $|a|bk + b|p|l = b$.

	Subbing in for $|a|b$: $|p|yk + b|p|l = b \implies |p|(xk +pl) = b$.

	So $|p||b$.
	
	Therefore $p|b$. 
\end{prf}

%%%%%%%%%%%%%%%%%%%%%%%%%%%%%%%%%%%%%%%%%%%%%%%%%%%%%%%%%%%%%%%%%%%%%%%%%%%%%%
\begin{proposition}
	If $A$, $B$, and $C$ are sets, then $A\cup(B\cap C) = (A\cup B)\cap(A\cup C)$.
	
\end{proposition}

\begin{prf}
	\phantom{ }

	Suppose $A$, $B$, and $C$ are sets.

	Consider the following sequence of equalities:

	\begin{adjustwidth}{2em}{0em}
		$A \cup (B \cap C) = \{x : (x \in A) \cup (B \cap C)\}$ \hfill (definition of set)

		$=\{x : (x \in A)\} \cup \{x: x \in B \land x \in C\}$ \hfill (definition of intersection)

		$=\{x : (x \in A) \lor (x \in B \land x \in C)\}$ \hfill (definition of union)

		$=\{x : (x \in A \lor x \in B) \land (x \in A \lor x \in C)\}$ \hfill (logical `or' distributive property)
	
		$=\{x : (x \in A \lor x \in B)\} \cap \{x: (x \in A \lor x \in C)\}$ \hfill (definition of intersection)
		
		$=(A \cup B) \cap \{x: (x \in A \lor x \in C)\}$ \hfill (definition of union)

		$=(A \cup B) \cap (A \cup C)$ \hfill (definition of union)
	
	\end{adjustwidth}

\end{prf}

\phantom{ }

\phantom{ }

\phantom{ }

\phantom{ }
%%%%%%%%%%%%%%%%%%%%%%%%%%%%%%%%%%%%%%%%%%%%%%%%%%%%%%%%%%%%%%%%%%%%%%%%%%%%%%
\begin{problem}
	Determine whether the following statement is true.
	If it is true, prove it; if it is false, give a disproof.
	\begin{center}
		If $X\subseteq A\cup B$, then $X\subseteq A$ or $X\subseteq B$.
	\end{center}
\end{problem}

\begin{soln}
	\phantom{ }

	Suppose $A = \{1\}$, $B = \{2\}$, and $X = \{1,2\}$.

	Then $A \cup B = \{1, 2\}$.

	So, $X = A \cup B$.

	Thus, $X \subseteq A \cup B$.

	But also, $X \nsubseteq A$ and $X \nsubseteq B$.
\end{soln}
%%%%%%%%%%%%%%%%%%%%%%%%%%%%%%%%%%%%%%%%%%%%%%%%%%%%%%%%%%%%%%%%%%%%%%%%%%%%%%

\end{document}