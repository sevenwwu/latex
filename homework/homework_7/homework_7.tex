 %%%%%%%%%%%%%%%%%%%%%%%%%%%%%%%%%%%%%%%%%%%%%%%%%
%%%%%%%%%%%%%%%%%%%%%%%%%%%%%%%%%%%%%%%%%%%%%%%%%
%%%      IGNORE THIS FIRST PART        %%%%%%%%%%   EXCEPT TO ENTER/DELETE YOUR NAME WHERE INDICATED
%%%%%%%%%%%%%%%%%%%%%%%%%%%%%%%%%%%%%%%%%%%%%%%%%
%%%%%%%%%%%%%%%%%%%%%%%%%%%%%%%%%%%%%%%%%%%%%%%%%

\documentclass[12pt]{amsart}
\usepackage{amsmath, amssymb, amsthm}
\usepackage{mathrsfs}
\usepackage{array}
\usepackage{enumitem}
\usepackage[usenames,dvipsnames]{xcolor}
\usepackage{graphicx}
\usepackage{fancyhdr}
\usepackage{caption}
\usepackage{hyperref}
\usepackage[labelformat=simple]{subcaption}
\usepackage[framemethod=default]{mdframed}
\usepackage{framed}
\usepackage{setspace}
\usepackage{changepage}

\setlength{\parindent}{0pt}

\newmdenv[linecolor=NavyBlue,backgroundcolor=White]{myframe}
\renewcommand\thesubfigure{(\Alph{subfigure})}
\renewcommand\thesubfigure{(\Alph{subfigure})}

\newcounter{problem_number}[section]
\newcommand{\num}{\refstepcounter{problem_number}\arabic{problem_number}}
\newcommand{\numlabel}[1]{\refstepcounter{problem_number}\label{#1}\arabic{problem_number}}

\newtheorem*{theorem}{Theorem}
\newtheoremstyle{named}{}{}{\itshape}{}{\bfseries}{.}{.5em}{\thmnote{#3}}
\theoremstyle{named}
\newtheorem*{namedtheorem}{Theorem}

\newenvironment{prf}
{\medskip\begin{color}{Gray}\begin{framed}\begin{color}{NavyBlue}\begin{proof}[Proof]
\doublespacing}
{\end{proof}\end{color}\end{framed}\end{color}\medskip}

\newenvironment{soln}
{\begin{color}{Gray}\begin{framed}\begin{color}{NavyBlue}\begin{proof}[Solution]
\doublespacing}
{\end{proof}\end{color}\end{framed}\end{color}}

\theoremstyle{definition}
\newtheorem{problem}{Problem}
\newtheorem{proposition}{Proposition}

\setenumerate[1]{label=(\roman*)}

\newcommand{\jeff}[1]{\textbf{\textcolor{WildStrawberry}{#1}}}
\newcommand{\student}[1]{\textbf{\textcolor{Orange}{#1}}}
\newcommand{\peer}[1]{\textbf{\textcolor{ForestGreen}{#1}}}

\textwidth=6.5in
\hoffset-.75in
\textheight=9in
\voffset-.75in
\footskip=30pt
\headheight=14pt
\pagestyle{fancy}
\lhead{\emph{\textcolor{Gray}{Homework \#7}}}			%%  UPDATE THE VERSION HERE!!!
\rhead{\emph{\textcolor{Gray}{student-name}}}			%%  ENTER/DELETE YOUR NAME HERE!!!
\chead{\emph{\textcolor{Gray}{MATH 309}}}
\cfoot{\thepage}
\renewcommand{\headrulewidth}{0.35pt}
\renewcommand{\footrulewidth}{0.35pt}
\thispagestyle{fancy}

%%%%%%%%%%%%%%%%%%%%%%%%%%%%%%%%%%%%%%%%%%%%%%%%%
%%%%%%%%%%%%%%%%%%%%%%%%%%%%%%%%%%%%%%%%%%%%%%%%%
%%%       ADD CUSTOM COMMANDS HERE     %%%%%%%%%%
%%%%%%%%%%%%%%%%%%%%%%%%%%%%%%%%%%%%%%%%%%%%%%%%%
%%%%%%%%%%%%%%%%%%%%%%%%%%%%%%%%%%%%%%%%%%%%%%%%%

\newcommand{\F}{\mathbb F}
\newcommand{\N}{\mathbb N}
\newcommand{\Q}{\mathbb Q}
\newcommand{\R}{\mathbb R}
\renewcommand{\S}{\mathbb S}
\newcommand{\Z}{\mathbb Z}
\newcommand{\RP}{\mathbb{RP}}

\newcommand{\Ll}{\mathcal L}
\newcommand{\Pp}{\mathcal P}
\newcommand{\Rr}{\mathcal R}

\newcommand{\Line}[1]{\overleftrightarrow{#1}}
\newcommand{\Ray}[1]{\overrightarrow{#1}}

%%%%%%%%%%%%%%%%%%%%%%%%%%%%%%%%%%%%%%%%%%%%%%%%%
%%%%%%%%%%%%%%%%%%%%%%%%%%%%%%%%%%%%%%%%%%%%%%%%%
%%%       NOW THE DOCUMENT BEGINS      %%%%%%%%%%
%%%%%%%%%%%%%%%%%%%%%%%%%%%%%%%%%%%%%%%%%%%%%%%%%
%%%%%%%%%%%%%%%%%%%%%%%%%%%%%%%%%%%%%%%%%%%%%%%%%

\begin{document}

Prove the following propositions. Format your proof so each step of the proof is on its own line; each line should still be a complete sentence. Below, I have entered a nonsensical proof as a model.

%%%%%%%%%%%%%%%%%%%%%%%%%%%%%%%%%%%%%%%%%%%%%%%%%%%%%%%%%%%%%%%%%%%%%%%%%%%%%%
\begin{proposition}
	Suppose $a,b,c\in\Z$.
	If $a^2+b^2=c^2$, then $a$ or $b$ is even.
\end{proposition}

\begin{prf}
	\phantom{ }

	% Lemma: All integers $a$ have the same parity as $a^2$.

	% \begin{adjustwidth}{2em}{0em} 
	% 	\textbf{Case 1:} $a$ is even.
		
	% 	\begin{adjustwidth}{2em}{0em} 
	% 		Then $a = 2x$ for some integer $x$.

	% 		Expression $a^2$ becomes $(2x)^2 = 4x^2 = 2(2x^2)$.
			
	% 		So $a^2 = 2y$ where $y$ is the integer $2x^2$.

	% 		Thus, $a^2$ is even.
	% 	\end{adjustwidth}

	% 	\textbf{Case 2:} $a$ is odd.

	% 	\begin{adjustwidth}{2em}{0em} 
	% 		Then $a = 2x + 1$ for some integer $x$.

	% 		Expression $a^2$ becomes $(2x+1)^2 = 4x^2+4x+1 = 2(2x^2+2x)+1$.

	% 		So $a^2 = 2y + 1$ where $y$ is the integer $2x^2+2x$.
			
	% 		Thus, $a^2$ is odd. 
	% 	\end{adjustwidth}
		
	% \end{adjustwidth}

	Suppose $a,b,c \in \mathbb Z$ such that $a^2 + b^2 = c^2$ and it's 
	not the case that $a$ or $b$ is even.

	Therefore, $a$ and $b$ are both odd.

	So $a = 2x + 1$ and $b = 2y + 1$ for some integers $x$ and $y$.

	Either $c$ is even or odd.

	\textbf{Case 1:} $c$ is even.
	\begin{adjustwidth}{2em}{0em} 
		Then $c = 2z$ for some integer $x$.

		The expression $a^2 + b^2 = c^2$ becomes $(2x + 1)^2 + (2y+1)^2 = 4z^2$.

		Expanding the expression yields $4x^2 + 4x + 1 + 4y^2 + 4y + 1 = 4z^2$.

		Factoring yields $4(x^2+y^2+x+y)+2 = 4z^2$.

		Simplifying the expression shows $4k + 2 = 4j$ where $k$ and $j$ are
		the integers $x^2+y^2+x+y$ and $4z^2$ respectfully. 

		Observe that $2 = 4(k - j)$, thus $4|2$.
		
		We have arrived at contradiction.
	\end{adjustwidth}

	\textbf{Case 2:} $c$ is odd.
	\begin{adjustwidth}{2em}{0em}
		Then $c = 2z + 1$ for some integer $x$.

		The expression $a^2 + b^2 = c^2$ becomes $(2x + 1)^2 + (2y+1)^2 = (2z+1)^2$.

		Expanded and factored yields $2(2x^2 + 2x + 2y^2 + 2y + 1) = 2(2z^2 + 2z) + 1$

		Simplifying the expression shows that $2k = 2j + 1$ for integers $k$ and $j$.

		Therefore, an even number equals an odd number. We have arrived at contradiction.
	\end{adjustwidth}
\end{prf}

\phantom{ }

\phantom{ }

\phantom{ }

\phantom{ }
%%%%%%%%%%%%%%%%%%%%%%%%%%%%%%%%%%%%%%%%%%%%%%%%%%%%%%%%%%%%%%%%%%%%%%%%%%%%%%
\begin{proposition}
	Suppose $x,y\in\Z$.
	If $x+y$ is even, then $x$ and $y$ have the same parity.
\end{proposition}

\begin{prf}
	
\end{prf}

%%%%%%%%%%%%%%%%%%%%%%%%%%%%%%%%%%%%%%%%%%%%%%%%%%%%%%%%%%%%%%%%%%%%%%%%%%%%%%
\begin{proposition}
	If $a\equiv b\pmod n$, then $\gcd(a,n) = \gcd(b,n)$.
\end{proposition}

\begin{prf}
		
\end{prf}
%%%%%%%%%%%%%%%%%%%%%%%%%%%%%%%%%%%%%%%%%%%%%%%%%%%%%%%%%%%%%%%%%%%%%%%%%%%%%%

\end{document}