%%%%%%%%%%%%%%%%%%%%%%%%%%%%%%%%%%%%%%%%%%%%%%%%%
%%%%%%%%%%%%%%%%%%%%%%%%%%%%%%%%%%%%%%%%%%%%%%%%%
%%%      IGNORE THIS FIRST PART        %%%%%%%%%%   EXCEPT TO ENTER/DELETE YOUR NAME WHERE INDICATED
%%%%%%%%%%%%%%%%%%%%%%%%%%%%%%%%%%%%%%%%%%%%%%%%%
%%%%%%%%%%%%%%%%%%%%%%%%%%%%%%%%%%%%%%%%%%%%%%%%%

\documentclass[12pt]{amsart}
\usepackage{amsmath, amssymb, amsthm}
\usepackage{mathrsfs}
\usepackage{array}
\usepackage{enumitem}
\usepackage[usenames,dvipsnames]{xcolor}
\usepackage{graphicx}
\usepackage{fancyhdr}
\usepackage{caption}
\usepackage{hyperref}
\usepackage[labelformat=simple]{subcaption}
\usepackage[framemethod=default]{mdframed}
\usepackage{framed}
\usepackage{setspace}
\usepackage{changepage}

\setlength{\parindent}{0pt}

\newmdenv[linecolor=NavyBlue,backgroundcolor=White]{myframe}
\renewcommand\thesubfigure{(\Alph{subfigure})}
\renewcommand\thesubfigure{(\Alph{subfigure})}

\newcounter{problem_number}[section]
\newcommand{\num}{\refstepcounter{problem_number}\arabic{problem_number}}
\newcommand{\numlabel}[1]{\refstepcounter{problem_number}\label{#1}\arabic{problem_number}}

\newtheorem*{theorem}{Theorem}
\newtheoremstyle{named}{}{}{\itshape}{}{\bfseries}{.}{.5em}{\thmnote{#3}}
\theoremstyle{named}
\newtheorem*{namedtheorem}{Theorem}

\newenvironment{prf}
{\medskip\begin{color}{Gray}\begin{framed}\begin{color}{NavyBlue}\begin{proof}[Proof]
\doublespacing}
{\end{proof}\end{color}\end{framed}\end{color}\medskip}

\newenvironment{soln}
{\begin{color}{Gray}\begin{framed}\begin{color}{NavyBlue}\begin{proof}[Solution]
\doublespacing}
{\end{proof}\end{color}\end{framed}\end{color}}

\theoremstyle{definition}
\newtheorem{proposition}{Proposition}
\newtheorem{problem}[proposition]{Problem}

\setenumerate[1]{label=(\roman*)}

\newcommand{\jeff}[1]{\textbf{\textcolor{WildStrawberry}{#1}}}
\newcommand{\student}[1]{\textbf{\textcolor{Orange}{#1}}}
\newcommand{\peer}[1]{\textbf{\textcolor{ForestGreen}{#1}}}

\textwidth=6.5in
\hoffset-.75in
\textheight=9in
\voffset-.75in
\footskip=30pt
\headheight=14pt
\pagestyle{fancy}
\lhead{\emph{\textcolor{Gray}{Homework \#9}}}			%%  UPDATE THE VERSION HERE!!!
\rhead{\emph{\textcolor{Gray}{Seven Lewis}}}			%%  ENTER/DELETE YOUR NAME HERE!!!
\chead{\emph{\textcolor{Gray}{MATH 309}}}
\cfoot{\thepage}
\renewcommand{\headrulewidth}{0.35pt}
\renewcommand{\footrulewidth}{0.35pt}
\thispagestyle{fancy}

%%%%%%%%%%%%%%%%%%%%%%%%%%%%%%%%%%%%%%%%%%%%%%%%%
%%%%%%%%%%%%%%%%%%%%%%%%%%%%%%%%%%%%%%%%%%%%%%%%%
%%%       ADD CUSTOM COMMANDS HERE     %%%%%%%%%%
%%%%%%%%%%%%%%%%%%%%%%%%%%%%%%%%%%%%%%%%%%%%%%%%%
%%%%%%%%%%%%%%%%%%%%%%%%%%%%%%%%%%%%%%%%%%%%%%%%%

\newcommand{\F}{\mathbb F}
\newcommand{\N}{\mathbb N}
\newcommand{\Q}{\mathbb Q}
\newcommand{\R}{\mathbb R}
\renewcommand{\S}{\mathbb S}
\newcommand{\Z}{\mathbb Z}
\newcommand{\RP}{\mathbb{RP}}

\newcommand{\Ll}{\mathcal L}
\newcommand{\Pp}{\mathcal P}
\newcommand{\Rr}{\mathcal R}

\newcommand{\Line}[1]{\overleftrightarrow{#1}}
\newcommand{\Ray}[1]{\overrightarrow{#1}}

%%%%%%%%%%%%%%%%%%%%%%%%%%%%%%%%%%%%%%%%%%%%%%%%%
%%%%%%%%%%%%%%%%%%%%%%%%%%%%%%%%%%%%%%%%%%%%%%%%%
%%%       NOW THE DOCUMENT BEGINS      %%%%%%%%%%
%%%%%%%%%%%%%%%%%%%%%%%%%%%%%%%%%%%%%%%%%%%%%%%%%
%%%%%%%%%%%%%%%%%%%%%%%%%%%%%%%%%%%%%%%%%%%%%%%%%

\begin{document}

Prove the following propositions. Format your proof so each step of the proof is on its own line; each line should still be a complete sentence.\\

%%%%%%%%%%%%%%%%%%%%%%%%%%%%%%%%%%%%%%%%%%%%%%%%%%%%%%%%%%%%%%%%%%%%%%%%%%%%%%
\begin{proposition}
	If $n\in\N$, then
	$$\frac{1}{2!}+\frac{2}{3!}+\frac{3}{4!}+\cdots+\frac{n}{(n+1)!} = 1 - \frac{1}{(n+1)!}.$$
	
\end{proposition}

\begin{prf}
	\phantom{ }

	Suppose $n \in \mathbb N$.

	\textbf{Basis Step:} $n = 1$
	\begin{adjustwidth}{2em}{0em}
		$\displaystyle\frac{1}{2!} = 1 - \frac{1}{(1)+1}$.

		\hspace*{1.3em}$\displaystyle = \frac{1}{2!}$ \hspace*{1em}\checkmark True
	\end{adjustwidth}

	\vspace*{0.5em}

	\textbf{Inductive Step:} $n = k + 1$
	\begin{adjustwidth}{2em}{0em}
		Assume $\displaystyle \frac{1}{2!}+\frac{2}{3!}+\frac{3}{4!}+\cdots+\frac{k}{(k+1)!} = 1 - \frac{1}{(k+1)!}$.
		
		Show that: 

		$\displaystyle 1 - \frac{1}{((k+1)+1)!} = \frac{1}{2!}+\frac{2}{3!}+\frac{3}{4!}+\cdots+\frac{(k+1)}{((k+1)+1)!}$

		\vspace*{0.5em}

		\hspace*{8.105em}$\displaystyle = \frac{1}{2!}+\frac{2}{3!}+\frac{3}{4!}+\cdots+\frac{k}{(k+1)!} +\frac{(k+1)}{((k+1)+1)!}$
		
		\vspace*{0.5em}
		
		\hspace*{8.105em}$\displaystyle = (1 - \frac{1}{(k+1)!}) +\frac{(k+1)}{((k+1)+1)!}$

		\vspace*{0.5em}

		\hspace*{8.105em}$\displaystyle = (1 - \frac{1}{(k+1)!}) +\frac{(k+1)}{(k+2)!}$

		\vspace*{0.5em}

		\hspace*{8.105em}$\displaystyle = 1 + (- \frac{1}{(k+1)!} \cdot \frac{k+2}{k+2}) +\frac{(k+1)}{(k+2)!}$

		\vspace*{0.5em}

		\hspace*{8.105em}$\displaystyle = 1 + (- \frac{k+2}{(k+2)!}) +\frac{(k+1)}{(k+2)!}$

		\vspace*{0.5em}

		\hspace*{8.105em}$\displaystyle = 1 + \frac{(k+1) - (k+2)}{(k+2)!}$

		\vspace*{0.5em}

		\hspace*{8.105em}$\displaystyle = 1 - \frac{1}{(k+2)!}$

		\vspace*{0.5em}

		\hspace*{8.105em}$\displaystyle = 1 - \frac{1}{((k+1)+1)!}$ \hspace*{1em}\checkmark True
	\end{adjustwidth}
\end{prf}

%%%%%%%%%%%%%%%%%%%%%%%%%%%%%%%%%%%%%%%%%%%%%%%%%%%%%%%%%%%%%%%%%%%%%%%%%%%%%%
\begin{problem}
	A chocolate bar consists of unit squares arranged in an $n\times m$ rectangular grid.
	You may split the bar into individual unit squares by breaking along the lines. What is the number of breaks required?
	Prove your answer is correct.
\end{problem}

\begin{soln}
	\phantom{ }

	\textbf{Proposition: } The number of breaks required is $nm - 1$. 

	Suppose $n,m \in \mathbb N$.

	\textbf{Basis Step:} $n = 1$
	\begin{adjustwidth}{2em}{0em}
		\textbf{Basis Step:} $m = 1$
		\begin{adjustwidth}{2em}{0em}
			The number of breaks required for an $1 \times 1$ square is trivially $0$ breaks.

			This is equal to $nm - 1 = (1)(1) - 1 = 0$. 
		\end{adjustwidth}

		\textbf{Inductive Step:} $m = k + 1$
		\begin{adjustwidth}{2em}{0em}
			Assume, for $m = k$, $nm - 1 = k - 1$ is the required number of breaks. 

			Show $nm - 1$ is the required number of breaks for $m = k + 1$.

			Suppose we have already performed the breaking for the first $k - 1$
			breaks. 
			
			At this point, when $m = k$, all the unit squares are separated. However
			$m = k + 1$, so we have one more unit square than previously. This means
			there is a unit square pair that has yet to be separated. So exactly $1$
			more split should occur.

			This is $(k - 1) + 1$ splits.

			$(k - 1) + 1 \implies n(k + 1) - 1$.
		\end{adjustwidth}

		So, when $n = 1$, $nm - 1$ is the number of splits required. 
	\end{adjustwidth}

	\textbf{Inductive Step:} $n = k + 1$
	\begin{adjustwidth}{2em}{0em}
		Assume, for $n = k$, $nm - 1 = km - 1$ is the required number of breaks.
		
		Show that $nm - 1$ is the required number of breaks for $n = k + 1$.

		Suppose we have already performed the break for the first $km - 1$ breaks. 

		At this point, when $n = k$, all the unit squares are separated. However
		$n = k + 1$, so we have $m$ more unit squares (because it's a $n \times m$
		rectangle). This will be bound in a line 
	\end{adjustwidth}
\end{soln}


%%%%%%%%%%%%%%%%%%%%%%%%%%%%%%%%%%%%%%%%%%%%%%%%%%%%%%%%%%%%%%%%%%%%%%%%%%%%%%
\begin{proposition}
	Let $n\in\N$ with $n\geq 2$, and let $A_1, A_2, \ldots, A_n$ be sets.
	Let $B$ be a set.
	Then,
	$$B\cap(A_1\cup A_2\cup \cdots\cup A_n) = (B\cap A_1)\cup (B\cap A_2) \cup \cdots \cup (B\cap A_n).$$
\end{proposition}

\begin{prf}
		
\end{prf}
%%%%%%%%%%%%%%%%%%%%%%%%%%%%%%%%%%%%%%%%%%%%%%%%%%%%%%%%%%%%%%%%%%%%%%%%%%%%%%


\end{document}