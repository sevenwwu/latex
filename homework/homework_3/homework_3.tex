%%%%%%%%%%%%%%%%%%%%%%%%%%%%%%%%%%%%%%%%%%%%%%%%%
%%%%%%%%%%%%%%%%%%%%%%%%%%%%%%%%%%%%%%%%%%%%%%%%%
%%%      IGNORE THIS FIRST PART        %%%%%%%%%%   EXCEPT TO ENTER/DELETE YOUR NAME WHERE INDICATED
%%%%%%%%%%%%%%%%%%%%%%%%%%%%%%%%%%%%%%%%%%%%%%%%%
%%%%%%%%%%%%%%%%%%%%%%%%%%%%%%%%%%%%%%%%%%%%%%%%%

\documentclass[12pt]{amsart}
\usepackage{amsmath, amssymb, amsthm}
\usepackage{mathrsfs}
\usepackage{array}
\usepackage{enumitem}
\usepackage[usenames,dvipsnames]{xcolor}
\usepackage{graphicx}
\usepackage{fancyhdr}
\usepackage{caption}
\usepackage{hyperref}
\usepackage[labelformat=simple]{subcaption}
\usepackage[framemethod=default]{mdframed}
\usepackage{framed}
\usepackage{setspace}
\usepackage{changepage}

\newmdenv[linecolor=NavyBlue,backgroundcolor=White]{myframe}
\renewcommand\thesubfigure{(\Alph{subfigure})}
\renewcommand\thesubfigure{(\Alph{subfigure})}

\newcounter{problem_number}[section]
\newcommand{\num}{\refstepcounter{problem_number}\arabic{problem_number}}
\newcommand{\numlabel}[1]{\refstepcounter{problem_number}\label{#1}\arabic{problem_number}}

\newtheorem*{theorem}{Theorem}
\newtheoremstyle{named}{}{}{\itshape}{}{\bfseries}{.}{.5em}{\thmnote{#3}}
\theoremstyle{named}
\newtheorem*{namedtheorem}{Theorem}

\newenvironment{prf}
{\medskip\begin{color}{Gray}\begin{framed}\begin{color}{NavyBlue}\begin{proof}[Proof]
\doublespacing}
{\end{proof}\end{color}\end{framed}\end{color}\medskip}

\newenvironment{soln}
{\begin{color}{Gray}\begin{framed}\begin{color}{NavyBlue}\begin{proof}[Solution]
\doublespacing}
{\end{proof}\end{color}\end{framed}\end{color}}

\theoremstyle{definition}
\newtheorem{problem}{Problem}

\setenumerate[1]{label=(\roman*)}

\newcommand{\jeff}[1]{\textbf{\textcolor{WildStrawberry}{#1}}}
\newcommand{\student}[1]{\textbf{\textcolor{Orange}{#1}}}
\newcommand{\peer}[1]{\textbf{\textcolor{ForestGreen}{#1}}}

\textwidth=6.5in
\hoffset-.75in
\textheight=9in
\voffset-.75in
\footskip=30pt
\headheight=14pt
\pagestyle{fancy}
\lhead{\emph{\textcolor{Gray}{Homework \#3}}}			%%  UPDATE THE VERSION HERE!!!
\rhead{\emph{\textcolor{Gray}{Seven Lewis}}}			%%  ENTER/DELETE YOUR NAME HERE!!!
\chead{\emph{\textcolor{Gray}{MATH 309}}}
\cfoot{\thepage}
\renewcommand{\headrulewidth}{0.35pt}
\renewcommand{\footrulewidth}{0.35pt}
\thispagestyle{fancy}

%%%%%%%%%%%%%%%%%%%%%%%%%%%%%%%%%%%%%%%%%%%%%%%%%
%%%%%%%%%%%%%%%%%%%%%%%%%%%%%%%%%%%%%%%%%%%%%%%%%
%%%       ADD CUSTOM COMMANDS HERE     %%%%%%%%%%
%%%%%%%%%%%%%%%%%%%%%%%%%%%%%%%%%%%%%%%%%%%%%%%%%
%%%%%%%%%%%%%%%%%%%%%%%%%%%%%%%%%%%%%%%%%%%%%%%%%

\newcommand{\F}{\mathbb F}
\newcommand{\N}{\mathbb N}
\newcommand{\Q}{\mathbb Q}
\newcommand{\R}{\mathbb R}
\renewcommand{\S}{\mathbb S}
\newcommand{\Z}{\mathbb Z}
\newcommand{\RP}{\mathbb{RP}}

\newcommand{\Ll}{\mathcal L}
\newcommand{\Pp}{\mathcal P}
\newcommand{\Rr}{\mathcal R}

\newcommand{\Line}[1]{\overleftrightarrow{#1}}
\newcommand{\Ray}[1]{\overrightarrow{#1}}

%%%%%%%%%%%%%%%%%%%%%%%%%%%%%%%%%%%%%%%%%%%%%%%%%
%%%%%%%%%%%%%%%%%%%%%%%%%%%%%%%%%%%%%%%%%%%%%%%%%
%%%       NOW THE DOCUMENT BEGINS      %%%%%%%%%%
%%%%%%%%%%%%%%%%%%%%%%%%%%%%%%%%%%%%%%%%%%%%%%%%%
%%%%%%%%%%%%%%%%%%%%%%%%%%%%%%%%%%%%%%%%%%%%%%%%%

\begin{document}

For these problems, you should justify your answers. You do not need to provide a rigorous mathematical proof, but rather an informal argument.

\vspace{5mm}



%%%%%%%%%%%%%%%%%%%%%%%%%%%%%%%%%%%%%%%%%%%%%%%%%%%%%%%%%%%%%%%%%%%%%%%%%%%%%%
\begin{problem}
	Determine whether the following statements are logically equivalent without using a truth table.
	$$P\wedge(Q\vee\sim Q)\ \text{ and }\ (\sim P)\Rightarrow (Q\wedge\sim Q)$$
\end{problem}

\begin{soln}
	\phantom{ }
    \begin{adjustwidth}{2em}{0em} 
        Let's first look at $P \land (Q \lor \sim Q)$:
        \begin{adjustwidth}{2em}{0em} 
            A logical AND is true only when both arguments are true.
            But $(Q \lor \sim Q)$ is always true, so this statement
            only depends upon the logical value of $P$. 
        \end{adjustwidth}

        \noindent Now $(\sim P)\Rightarrow (Q\wedge\sim Q)$:
        \begin{adjustwidth}{2em}{0em}
            Because $(Q \lor \sim Q)$ is always false, the implication
            statement will be true when $\sim P$ is false and false when
            $\sim P$ is true. This is the same as saying it is logically equivalent
            to $P$. 
        \end{adjustwidth}

        \noindent Because both statements are logically equivalent to $P$, and $P$ is
        logically equivalent to itself. Thus, they are logically equivalent.
    \end{adjustwidth}
    \phantom{ }

    \phantom{ }

    \phantom{ }

    \phantom{ }

    \phantom{ }

    \phantom{ }

    \phantom{ }

    \phantom{ }

    \phantom{ }

    \phantom{ }

    \phantom{ }

    \phantom{ }

    \phantom{ }
\end{soln}

%%%%%%%%%%%%%%%%%%%%%%%%%%%%%%%%%%%%%%%%%%%%%%%%%%%%%%%%%%%%%%%%%%%%%%%%%%%%%%
\begin{problem}
	Complete the following truth table.
\end{problem}

$$
\renewcommand{\arraystretch}{2}
\begin{array}{|c|c|c||c|c|c|}
\hline
P&Q&R&\ \sim Q \  &\ \sim Q\vee R\  &\ P\Rightarrow (\sim Q\vee R) \\
\hline\hline
\textsc{t}&\textsc{t}&\textsc{t}& \textsc{f}& \textsc{t}& \textsc{t}\\
\hline
\textsc{t}&\textsc{t}&\textsc{f}& \textsc{f}& \textsc{f}& \textsc{f}\\
\hline
\textsc{t}&\textsc{f}&\textsc{t}& \textsc{t}& \textsc{t}& \textsc{t}\\
\hline
\textsc{t}&\textsc{f}&\textsc{f}& \textsc{t}& \textsc{t}& \textsc{t}\\
\hline
\textsc{f}&\textsc{t}&\textsc{t}& \textsc{f}& \textsc{t}& \textsc{t}\\
\hline
\textsc{f}&\textsc{t}&\textsc{f}& \textsc{f}& \textsc{f}& \textsc{t}\\
\hline
\textsc{f}&\textsc{f}&\textsc{t}& \textsc{t}& \textsc{t}& \textsc{t}\\
\hline
\textsc{f}&\textsc{f}&\textsc{f}& \textsc{t}& \textsc{t}& \textsc{t}\\
\hline
\end{array}
$$


\newpage
%%%%%%%%%%%%%%%%%%%%%%%%%%%%%%%%%%%%%%%%%%%%%%%%%%%%%%%%%%%%%%%%%%%%%%%%%%%%%%

\begin{problem}
	Use a truth table to verify the distributive law:
	$$P\wedge(Q\vee R) = (P\wedge Q)\vee(P\wedge R)$$
\end{problem}

$$
\renewcommand{\arraystretch}{2}
\begin{array}{|c|c|c||c|c|c|c|}
\hline
P&Q&R& Q \lor R & P\wedge(Q\vee R) &(P\wedge Q)\vee(P\wedge R)\\
\hline\hline
\textsc{t}&\textsc{t}&\textsc{t}& \textsc{t}& \textsc{t}& \textsc{t}\\
\hline
\textsc{t}&\textsc{t}&\textsc{f}& \textsc{t}& \textsc{t}& \textsc{t}\\
\hline
\textsc{t}&\textsc{f}&\textsc{t}& \textsc{t}& \textsc{t}& \textsc{t}\\
\hline
\textsc{t}&\textsc{f}&\textsc{f}& \textsc{f}& \textsc{f}& \textsc{f}\\
\hline
\textsc{f}&\textsc{t}&\textsc{t}& \textsc{t}& \textsc{f}& \textsc{f}\\
\hline
\textsc{f}&\textsc{t}&\textsc{f}& \textsc{t}& \textsc{f}& \textsc{f}\\
\hline
\textsc{f}&\textsc{f}&\textsc{t}& \textsc{t}& \textsc{f}& \textsc{f}\\
\hline
\textsc{f}&\textsc{f}&\textsc{f}& \textsc{f}& \textsc{f}& \textsc{f}\\
\hline
\end{array}
$$

%%%%%%%%%%%%%%%%%%%%%%%%%%%%%%%%%%%%%%%%%%%%%%%%%%%%%%%%%%%%%%%%%%%%%%%%%%%%%%
\begin{problem}
	Use a truth table to verify the following is a \textbf{tautology} -- i.e., it is \emph{always} true.
	$$[(P\Rightarrow Q)\wedge(Q\Rightarrow R)]\Rightarrow (P\Rightarrow R)$$
\end{problem}

$$
\renewcommand{\arraystretch}{2}
\begin{array}{|c|c|c||c|c|c|c|}
\hline
P&Q&R&P\Rightarrow Q&Q\Rightarrow R&P\Rightarrow R& [(P\Rightarrow Q)\wedge(Q\Rightarrow R)]\Rightarrow (P\Rightarrow R)\\
\hline\hline
\textsc{t}&\textsc{t}&\textsc{t}&\textsc{t} &\textsc{t} &\textsc{t} &\textsc{t} \\
\hline
\textsc{t}&\textsc{t}&\textsc{f}&\textsc{t} &\textsc{f} &\textsc{f} &\textsc{t} \\
\hline
\textsc{t}&\textsc{f}&\textsc{t}&\textsc{f} &\textsc{t} &\textsc{t} &\textsc{t} \\
\hline
\textsc{t}&\textsc{f}&\textsc{f}&\textsc{f} &\textsc{t} &\textsc{f} &\textsc{t} \\
\hline
\textsc{f}&\textsc{t}&\textsc{t}&\textsc{t} &\textsc{t} &\textsc{t} &\textsc{t} \\
\hline
\textsc{f}&\textsc{t}&\textsc{f}&\textsc{t} &\textsc{f} &\textsc{t} &\textsc{t} \\
\hline
\textsc{f}&\textsc{f}&\textsc{t}&\textsc{t} &\textsc{t} &\textsc{t} &\textsc{t} \\
\hline
\textsc{f}&\textsc{f}&\textsc{f}&\textsc{t} &\textsc{t} &\textsc{t} &\textsc{t} \\
\hline
\end{array}
$$


%%%%%%%%%%%%%%%%%%%%%%%%%%%%%%%%%%%%%%%%%%%%%%%%%%%%%%%%%%%%%%%%%%%%%%%%%%%%%%
\end{document}