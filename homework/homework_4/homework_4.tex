%%%%%%%%%%%%%%%%%%%%%%%%%%%%%%%%%%%%%%%%%%%%%%%%%
%%%%%%%%%%%%%%%%%%%%%%%%%%%%%%%%%%%%%%%%%%%%%%%%%
%%%      IGNORE THIS FIRST PART        %%%%%%%%%%   EXCEPT TO ENTER/DELETE YOUR NAME WHERE INDICATED
%%%%%%%%%%%%%%%%%%%%%%%%%%%%%%%%%%%%%%%%%%%%%%%%%
%%%%%%%%%%%%%%%%%%%%%%%%%%%%%%%%%%%%%%%%%%%%%%%%%

\documentclass[12pt]{amsart}
\usepackage{amsmath, amssymb, amsthm}
\usepackage{mathrsfs}
\usepackage{array}
\usepackage{enumitem}
\usepackage[usenames,dvipsnames]{xcolor}
\usepackage{graphicx}
\usepackage{fancyhdr}
\usepackage{caption}
\usepackage{hyperref}
\usepackage[labelformat=simple]{subcaption}
\usepackage[framemethod=default]{mdframed}
\usepackage{framed}
\usepackage{setspace}
\newmdenv[linecolor=NavyBlue,backgroundcolor=White]{myframe}
\renewcommand\thesubfigure{(\Alph{subfigure})}
\renewcommand\thesubfigure{(\Alph{subfigure})}

\newcounter{problem_number}[section]
\newcommand{\num}{\refstepcounter{problem_number}\arabic{problem_number}}
\newcommand{\numlabel}[1]{\refstepcounter{problem_number}\label{#1}\arabic{problem_number}}

\newtheorem*{theorem}{Theorem}
\newtheoremstyle{named}{}{}{\itshape}{}{\bfseries}{.}{.5em}{\thmnote{#3}}
\theoremstyle{named}
\newtheorem*{namedtheorem}{Theorem}

\newenvironment{prf}
{\medskip\begin{color}{Gray}\begin{framed}\begin{color}{NavyBlue}\begin{proof}[Proof]
\doublespacing}
{\end{proof}\end{color}\end{framed}\end{color}\medskip}

\newenvironment{soln}
{\begin{color}{Gray}\begin{framed}\begin{color}{NavyBlue}\begin{proof}[Solution]
\doublespacing}
{\end{proof}\end{color}\end{framed}\end{color}}

\theoremstyle{definition}
\newtheorem{problem}{Problem}

\setenumerate[1]{label=(\roman*)}

\newcommand{\jeff}[1]{\textbf{\textcolor{WildStrawberry}{#1}}}
\newcommand{\student}[1]{\textbf{\textcolor{Orange}{#1}}}
\newcommand{\peer}[1]{\textbf{\textcolor{ForestGreen}{#1}}}

\textwidth=6.5in
\hoffset-.75in
\textheight=9in
\voffset-.75in
\footskip=30pt
\headheight=14pt
\pagestyle{fancy}
\lhead{\emph{\textcolor{Gray}{Homework \#4}}}			%%  UPDATE THE VERSION HERE!!!
\rhead{\emph{\textcolor{Gray}{student-name}}}			%%  ENTER/DELETE YOUR NAME HERE!!!
\chead{\emph{\textcolor{Gray}{MATH 309}}}
\cfoot{\thepage}
\renewcommand{\headrulewidth}{0.35pt}
\renewcommand{\footrulewidth}{0.35pt}
\thispagestyle{fancy}

%%%%%%%%%%%%%%%%%%%%%%%%%%%%%%%%%%%%%%%%%%%%%%%%%
%%%%%%%%%%%%%%%%%%%%%%%%%%%%%%%%%%%%%%%%%%%%%%%%%
%%%       ADD CUSTOM COMMANDS HERE     %%%%%%%%%%
%%%%%%%%%%%%%%%%%%%%%%%%%%%%%%%%%%%%%%%%%%%%%%%%%
%%%%%%%%%%%%%%%%%%%%%%%%%%%%%%%%%%%%%%%%%%%%%%%%%

\newcommand{\F}{\mathbb F}
\newcommand{\N}{\mathbb N}
\newcommand{\Q}{\mathbb Q}
\newcommand{\R}{\mathbb R}
\renewcommand{\S}{\mathbb S}
\newcommand{\Z}{\mathbb Z}
\newcommand{\RP}{\mathbb{RP}}

\newcommand{\Ll}{\mathcal L}
\newcommand{\Pp}{\mathcal P}
\newcommand{\Rr}{\mathcal R}

\newcommand{\Line}[1]{\overleftrightarrow{#1}}
\newcommand{\Ray}[1]{\overrightarrow{#1}}

%%%%%%%%%%%%%%%%%%%%%%%%%%%%%%%%%%%%%%%%%%%%%%%%%
%%%%%%%%%%%%%%%%%%%%%%%%%%%%%%%%%%%%%%%%%%%%%%%%%
%%%       NOW THE DOCUMENT BEGINS      %%%%%%%%%%
%%%%%%%%%%%%%%%%%%%%%%%%%%%%%%%%%%%%%%%%%%%%%%%%%
%%%%%%%%%%%%%%%%%%%%%%%%%%%%%%%%%%%%%%%%%%%%%%%%%

\begin{document}

For these problems, you should justify your answers. You do not need to provide a rigorous mathematical proof, but rather an informal argument.

\vspace{5mm}

%%%%%%%%%%%%%%%%%%%%%%%%%%%%%%%%%%%%%%%%%%%%%%%%%%%%%%%%%%%%%%%%%%%%%%%%%%%%%%
\begin{problem}
	Symbolic logic can be used to find new expressions that are equivalent to old ones.
	\begin{enumerate}
		\item Find an expression that is logically equivalent to the biconditional $P\Leftrightarrow Q$ that doesn't use $\Leftrightarrow$, $\Rightarrow$, or $\Leftarrow$.
		\item Find an expression that is logically equivalent to the conditional $P\Rightarrow Q$ using only $\wedge$, $\vee$, and $\sim$.
		\item Can you express $P\wedge Q$ using only $\vee$ and $\sim$? Justify your answer.
		\item Can you express $P\vee Q$ using only $\wedge$ and $\sim$? Justify your answer.
	\end{enumerate}
	This exercise shows that some of the symbols we use are redundant, but some are not. In any event, they are all useful.
\end{problem}

\begin{soln}
	
\end{soln}
%%%%%%%%%%%%%%%%%%%%%%%%%%%%%%%%%%%%%%%%%%%%%%%%%%%%%%%%%%%%%%%%%%%%%%%%%%%%%%
\begin{problem}
	Consider the following statement.
$$\forall N\in\N,\exists X\in\mathscr P(\N),|X|\geq N$$

	\begin{enumerate}
		\item Write the statement as an English sentence.
		\item Give the negation of the statement in symbolic logic. (You answer should have no $\sim$ symbols.)
		\item Write the negation of the statement as an English sentence.
		\item Is the original statement true or false? Justify your answer.
	\end{enumerate}
\end{problem}

\begin{soln}
	
\end{soln}
%%%%%%%%%%%%%%%%%%%%%%%%%%%%%%%%%%%%%%%%%%%%%%%%%%%%%%%%%%%%%%%%%%%%%%%%%%%%%%
\begin{problem}
	Consider the following English sentence.
\begin{center}
	\emph{If $r$ is a rational number and $r\not=0$, then $\frac{M}{r}$ is an integer for some natural number $M$.}
\end{center}

	\begin{enumerate}
		\item Write the statement using symbolic logic.
		\item Give the negation of the statement in symbolic logic. (You answer should have no $\sim$ symbols.)
		\item Write the negation of the statement as an English sentence.
		\item Is the original statement true or false? Justify your answer.
	\end{enumerate}
\end{problem}

\begin{soln}
	
\end{soln}


%%%%%%%%%%%%%%%%%%%%%%%%%%%%%%%%%%%%%%%%%%%%%%%%%%%%%%%%%%%%%%%%%%%%%%%%%%%%%%
\end{document}