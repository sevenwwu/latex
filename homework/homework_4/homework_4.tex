%%%%%%%%%%%%%%%%%%%%%%%%%%%%%%%%%%%%%%%%%%%%%%%%%
%%%%%%%%%%%%%%%%%%%%%%%%%%%%%%%%%%%%%%%%%%%%%%%%%
%%%      IGNORE THIS FIRST PART        %%%%%%%%%%   EXCEPT TO ENTER/DELETE YOUR NAME WHERE INDICATED
%%%%%%%%%%%%%%%%%%%%%%%%%%%%%%%%%%%%%%%%%%%%%%%%%
%%%%%%%%%%%%%%%%%%%%%%%%%%%%%%%%%%%%%%%%%%%%%%%%%

\documentclass[12pt]{amsart}
\usepackage{amsmath, amssymb, amsthm}
\usepackage{mathrsfs}
\usepackage{array}
\usepackage{enumitem}
\usepackage[usenames,dvipsnames]{xcolor}
\usepackage{graphicx}
\usepackage{fancyhdr}
\usepackage{caption}
\usepackage{hyperref}
\usepackage[labelformat=simple]{subcaption}
\usepackage[framemethod=default]{mdframed}
\usepackage{framed}
\usepackage{setspace}
\usepackage{changepage}

\newmdenv[linecolor=NavyBlue,backgroundcolor=White]{myframe}
\renewcommand\thesubfigure{(\Alph{subfigure})}
\renewcommand\thesubfigure{(\Alph{subfigure})}

\newcounter{problem_number}[section]
\newcommand{\num}{\refstepcounter{problem_number}\arabic{problem_number}}
\newcommand{\numlabel}[1]{\refstepcounter{problem_number}\label{#1}\arabic{problem_number}}

\newtheorem*{theorem}{Theorem}
\newtheoremstyle{named}{}{}{\itshape}{}{\bfseries}{.}{.5em}{\thmnote{#3}}
\theoremstyle{named}
\newtheorem*{namedtheorem}{Theorem}

\newenvironment{prf}
{\medskip\begin{color}{Gray}\begin{framed}\begin{color}{NavyBlue}\begin{proof}[Proof]
\doublespacing}
{\end{proof}\end{color}\end{framed}\end{color}\medskip}

\newenvironment{soln}
{\begin{color}{Gray}\begin{framed}\begin{color}{NavyBlue}\begin{proof}[Solution]
\doublespacing}
{\end{proof}\end{color}\end{framed}\end{color}}

\theoremstyle{definition}
\newtheorem{problem}{Problem}

\setenumerate[1]{label=(\roman*)}

\newcommand{\jeff}[1]{\textbf{\textcolor{WildStrawberry}{#1}}}
\newcommand{\student}[1]{\textbf{\textcolor{Orange}{#1}}}
\newcommand{\peer}[1]{\textbf{\textcolor{ForestGreen}{#1}}}

\textwidth=6.5in
\hoffset-.75in
\textheight=9in
\voffset-.75in
\footskip=30pt
\headheight=14pt
\pagestyle{fancy}
\lhead{\emph{\textcolor{Gray}{Homework \#4}}}			%%  UPDATE THE VERSION HERE!!!
\rhead{\emph{\textcolor{Gray}{Seven Lewis}}}			%%  ENTER/DELETE YOUR NAME HERE!!!
\chead{\emph{\textcolor{Gray}{MATH 309}}}
\cfoot{\thepage}
\renewcommand{\headrulewidth}{0.35pt}
\renewcommand{\footrulewidth}{0.35pt}
\thispagestyle{fancy}

%%%%%%%%%%%%%%%%%%%%%%%%%%%%%%%%%%%%%%%%%%%%%%%%%
%%%%%%%%%%%%%%%%%%%%%%%%%%%%%%%%%%%%%%%%%%%%%%%%%
%%%       ADD CUSTOM COMMANDS HERE     %%%%%%%%%%
%%%%%%%%%%%%%%%%%%%%%%%%%%%%%%%%%%%%%%%%%%%%%%%%%
%%%%%%%%%%%%%%%%%%%%%%%%%%%%%%%%%%%%%%%%%%%%%%%%%

\newcommand{\F}{\mathbb F}
\newcommand{\N}{\mathbb N}
\newcommand{\Q}{\mathbb Q}
\newcommand{\R}{\mathbb R}
\renewcommand{\S}{\mathbb S}
\newcommand{\Z}{\mathbb Z}
\newcommand{\RP}{\mathbb{RP}}

\newcommand{\Ll}{\mathcal L}
\newcommand{\Pp}{\mathcal P}
\newcommand{\Rr}{\mathcal R}

\newcommand{\Line}[1]{\overleftrightarrow{#1}}
\newcommand{\Ray}[1]{\overrightarrow{#1}}

%%%%%%%%%%%%%%%%%%%%%%%%%%%%%%%%%%%%%%%%%%%%%%%%%
%%%%%%%%%%%%%%%%%%%%%%%%%%%%%%%%%%%%%%%%%%%%%%%%%
%%%       NOW THE DOCUMENT BEGINS      %%%%%%%%%%
%%%%%%%%%%%%%%%%%%%%%%%%%%%%%%%%%%%%%%%%%%%%%%%%%
%%%%%%%%%%%%%%%%%%%%%%%%%%%%%%%%%%%%%%%%%%%%%%%%%

\begin{document}

For these problems, you should justify your answers. You do not need to provide a rigorous mathematical proof, but rather an informal argument.

\vspace{5mm}

%%%%%%%%%%%%%%%%%%%%%%%%%%%%%%%%%%%%%%%%%%%%%%%%%%%%%%%%%%%%%%%%%%%%%%%%%%%%%%
\begin{problem}
	Symbolic logic can be used to find new expressions that are equivalent to old ones.
	\begin{enumerate}
		\item Find an expression that is logically equivalent to the biconditional $P\Leftrightarrow Q$ that doesn't use $\Leftrightarrow$, $\Rightarrow$, or $\Leftarrow$.
		\item Find an expression that is logically equivalent to the conditional $P\Rightarrow Q$ using only $\wedge$, $\vee$, and $\sim$.
		\item Can you express $P\wedge Q$ using only $\vee$ and $\sim$? Justify your answer.
		\item Can you express $P\vee Q$ using only $\wedge$ and $\sim$? Justify your answer.
	\end{enumerate}
	This exercise shows that some of the symbols we use are redundant, but some are not. In any event, they are all useful.
\end{problem}

\begin{soln}
	\phantom{ }
	\begin{enumerate}
		\item $(Q \land P) \lor (\sim P \land \sim Q)$
		
		\noindent Either both $P$ and $Q$, or not $P$ and not $Q$ (but not both because
		that would lead to contradiction). This is really what $P \Leftrightarrow Q$ is 
		saying.

		\phantom{ }



		\item $\sim P \lor Q$
		
		\noindent This is material implication and is actually the definition of a 
		logical conditional statement. It's derived from the only case when a 
		conditional is false: 
		
		\noindent $\sim (P \land \sim Q) \iff (\sim P \lor Q)$

		\phantom{ }



		\item $\sim (\sim P \lor \sim Q)$
		
		\noindent Applying De Morgan's law and double negation elimination: 
		
		\noindent $\sim (\sim P \lor \sim Q) \iff (\sim \sim P \land \sim \sim Q)
		\iff (P \land Q)$

		\phantom{ }


		
		\item $\sim (\sim P \land \sim Q)$
		
		\noindent De Morgan's law is true inversely as well, so the same argument
		applies:
		
		\noindent $\sim (\sim P \land \sim Q) \iff (\sim \sim P \lor \sim \sim Q)
		\iff (P \lor Q)$
	\end{enumerate}
\end{soln}
%%%%%%%%%%%%%%%%%%%%%%%%%%%%%%%%%%%%%%%%%%%%%%%%%%%%%%%%%%%%%%%%%%%%%%%%%%%%%%
\begin{problem}
	Consider the following statement.
$$\forall N\in\N,\exists X\in\mathscr P(\N),|X|\geq N$$

	\begin{enumerate}
		\item Write the statement as an English sentence.
		\item Give the negation of the statement in symbolic logic. (You answer should have no $\sim$ symbols.)
		\item Write the negation of the statement as an English sentence.
		\item Is the original statement true or false? Justify your answer.
	\end{enumerate}
\end{problem}

\begin{soln}
	\phantom{ }

	\begin{enumerate}
		\item ``For all natural numbers, there is at least one subset of the 
		natural numbers that has a cardinality greater than or equal to that
		particular natural number.''
		

		\item $\sim (\forall N\in\N,\exists X\in\mathscr P(\N),|X|\geq N)$
		
		\noindent We can start by nesting the quantifiers to make simplification clearer:
		
		\noindent $\sim \forall N ((N \in \mathbb N) \Rightarrow (\exists X((X \in \mathscr P(\N)) \land (|X| \geq N))))$
		
		\noindent Then we can move the negation inward: 

		\noindent $\exists N \sim ((N \in \mathbb N) \Rightarrow (\exists X((X \in \mathscr P(\N)) \land (|X| \geq N))))$

		\noindent $\exists N ((N \in \mathbb N) \land \sim (\exists X((X \in \mathscr P(\N)) \land (|X| \geq N))))$

		\noindent $\exists N ((N \in \mathbb N) \land (\forall X \sim((X \in \mathscr P(\N)) \land (|X| \geq N))))$

		\noindent $\exists N ((N \in \mathbb N) \land (\forall X ((X \in \mathscr P(\N)) \Rightarrow \sim (|X| \geq N))))$

		\noindent $\exists N ((N \in \mathbb N) \land (\forall X ((X \in \mathscr P(\N)) \Rightarrow (|X| < N))))$

		\noindent De-nest quantifiers:

		\noindent $\exists N \in \mathbb N, \forall X \in \mathscr P (\mathbb N),N > |X|$


		\item ``There exists a natural number that is greater than cardinality of every subset
		of the natural numbers.''

		\item The original statement is true (and its negation is false). This
		is because for each natural number, you can make a subset that contains
		all proceeding numbers and the number in question. The cardinality of
		this set will match the value of the number. 

		\phantom{ }

	\end{enumerate}
\end{soln}


%%%%%%%%%%%%%%%%%%%%%%%%%%%%%%%%%%%%%%%%%%%%%%%%%%%%%%%%%%%%%%%%%%%%%%%%%%%%%%
\begin{problem}
	Consider the following English sentence.
\begin{center}
	\emph{If $r$ is a rational number and $r\not=0$, then $\frac{M}{r}$ is an integer for some natural number $M$.}
\end{center}

	\begin{enumerate}
		\item Write the statement using symbolic logic.
		\item Give the negation of the statement in symbolic logic. (You answer should have no $\sim$ symbols.)
		\item Write the negation of the statement as an English sentence.
		\item Is the original statement true or false? Justify your answer.
	\end{enumerate}
\end{problem}

\begin{soln}
	\phantom{ }
	\begin{enumerate}
		\item $\forall r \in \mathbb Q,\exists M \in \mathbb N,
		(r \neq 0) \Rightarrow  (\frac{M}{r} \in \mathbb Z)$

		\item $\exists r \in \mathbb Q, \forall M \in \mathbb N, (r \neq 0)
		\land (\frac{M}{r} \notin \mathbb Z)$

		\item ``There exists some rational number such that there is no natural number
		it can divide and yield an integer.''

		\item The original statement is true (and its negation false). 
		
		% \noindent $\forall r \in \mathbb Q$ can be thought of as 
		% $\forall r \in \{\frac{a}{b}:a,b\in \mathbb Z, b \neq 0\}$.
		
		% \noindent Is there a natural number for every combo $a$ and $b$ such that 
		% $\frac{M}{\frac{a}{b}}$ is an integer? 
		
		% \noindent Yes, because $\frac{M}{\frac{a}{b}} 
		% = \frac{M}{a}b$ and $a \neq 0$, we can pick $M = |a|$ for every rational
		% number. $\pm b \in \mathbb Z$.

		\noindent Suppose $a,b \in \mathbb Z$, where $b \neq 0$, such that $\frac{a}{b} = r$.

		\noindent $a \neq 0$ because $r \neq 0$.

		\noindent Let $M$ be some natural number equal to $|a|$.
		
		\noindent $\frac{M}{\frac{a}{b}} = \frac{M}{a}b = \frac{|a|}{a}b = \pm b$

		\noindent Thus, $\pm b \in \mathbb Z$ for all $a$ and $b$, where $a \neq 0$
		and $b \neq 0$. 
	\end{enumerate}
\end{soln}


%%%%%%%%%%%%%%%%%%%%%%%%%%%%%%%%%%%%%%%%%%%%%%%%%%%%%%%%%%%%%%%%%%%%%%%%%%%%%%
\end{document}